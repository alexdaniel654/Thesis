\chapter{Automated Segmentation of Kidneys using Machine Learning}
\label{chap:ML}

\begin{abstract}
	\ac{TKV} is an important measure in renal disease detection and monitoring. Here a fully automated method to segment the kidneys from \ttwo-weighted \ac{MRI} to calculate \ac{TKV} of \ac{HC} and \ac{CKD} patients is developed.
	
	This automated method uses machine learning, specifically a 2D \ac{CNN}, to accurately segment the left and right kidneys from \ttwo-weighted \ac{MRI} data. The dataset consisted of 30 \ac{HC} subjects and 30 \ac{CKD} patients. The model was trained on 50 manually defined \ac{HC} and \ac{CKD} kidney segmentations. It was subsequently evaluated on 50 test data sets, comprising data from five \ac{HC}s and five \ac{CKD} patients each scanned five times in a scan session to enable comparison of the precision of the \ac{CNN} and manual segmentation of kidneys.
	
	The unseen test data processed by the 2D \ac{CNN} had a mean Dice score of 0.93 $\pm$ 0.01. The difference between manual and automatically computed \ac{TKV} was 1.2 $\pm$ 16.2 m$\ell$ with a mean surface distance of 0.65 $\pm$ 0.21 mm. The variance in \ac{TKV} measurements from repeat acquisitions on the same subject was significantly lower using the automated method compared to manual segmentation of the kidneys.
	
	The 2D \ac{CNN} method provides fully automated segmentation of the left and right kidney and calculation of \ac{TKV} in under ten seconds on a standard office computer, allowing high data throughput and is a freely available executable.
	
	This work was presented as an aural presentation at the \ac{ISMRM} 28th Annual Meeting (2020) \cite{daniel_automated_2020}.
\end{abstract}
\newpage
\acresetall
\section{Introduction}

Segmentation of the kidneys from \ac{MRI} is a time consuming aspect of many renal \ac{MRI} studies \cite{cox_multiparametric_2017, cohen_mri_2009, van_den_dool_functional_2005}. \ac{TKV} gives insight into renal function and is therefore used as a measured parameter for a variety of renal pathologies. The use of \ac{TKV} is an active area of ongoing research for \ac{ADPKD}, which is characterised by an increase in \ac{TKV} due to cyst formation. Disease progression can be monitored by recording \ac{TKV}, with higher rates of \ac{TKV} increase being associated with a more rapid decrease in renal function \cite{chapman_kidney_2012, tangri_total_2017, grantham_volume_2006}. Measurements of \ac{TKV} in \ac{CKD} subjects have shown a significant correlation with glomerular filtration rate \cite{buchanan_quantitative_2019}, the primary measure of \ac{CKD} severity \cite{stevens_assessing_2006}, with more generally a decrease in \ac{TKV} associated with a decrease in renal function \cite{gong_relationship_2012}. When studying pathologies which commonly lead to a change in kidney function, total kidney perfusion is often measured, this metric relies on an accurate measurement of renal blood flow and kidney volume of each kidney, and allows investigators to ascertain if the blood flow is preserved as the organ changes in size or if tissue perfusion is impaired. In addition to \ac{TKV} measurements, renal segmentation is an important first step for many other processing pipelines, be that for automated cortical-medullary segmentations or to carry out multiparametric mapping within only the kidney to reduce computation times. 

The gold standards of kidney segmentation are manual \ac{ROI} boundary tracing \cite{di_leo_measurement_2011} or stereology \cite{bae_volumetric_2000} by experienced and skilled experts, with blood vessels in the kidney and the hilum excluded. These manual processes are highly time consuming (taking approximately 15 – 30 minutes per subject \cite{zollner_assessment_2012, sharma_kidney_2017, simms_rapid_2019} and can be biased by investigator judgement due to the similar signal intensities between the kidneys and surrounding organs, anatomical differences between subjects, cysts and image artefacts. Consequently, the resulting kidney \ac{ROI}s produced are subject to intra- and inter-expert variability as a result of the varying expertise levels; experts may segment a specific image differently when performed more than once, or different experts may segment the same image differently. These factors mean that the development of a faster and ideally fully automated method of renal segmentation is highly desirable. However the same factors that make manual segmentation difficult can also limit fully automated methods, for example the signal intensity of the kidneys closely matches that of other abdominal structures such as the spleen.
\newpage
A number of automated methods have been proposed with varied success \cite{zollner_assessment_2012}. Some simply assume the kidney is an ellipse and calculate the volume from measurements of the pole-to-pole distance \cite{cheong_normal_2007, spithoven_estimation_2015} or include a correction factor to reduce overestimations \cite{seuss_development_2017}. Unfortunately these techniques produce a large confidence interval and still require human intervention to define the pole-to-pole length, a process that can produce inconsistencies between readers and takes a reasonable amount of time ($\approx$5 min) \cite{magistroni_review_2018}. Other semi-automated methods use classical image processing techniques such as thresholding \cite{coulam_measurement_2002}, water-shedding \cite{karstoft_different_2007}, level sets \cite{simms_rapid_2019, gloger_prior_2012}, and spatial prior probability mapping \cite{kim_automated_2016}. These methods can either be inaccurate, over-segmenting the kidneys, or include a number of parameters that need to be manually adjusted and are computationally intensive. Further, the fact that each technique is highly optimised for a specific dataset means that it needs to be re-written to be applied to different pathology, another time consuming and highly skilled process.

Machine learning methods have the potential to automatically detect different patterns from data given to a model which has been trained. Deep learning is a class of machine learning algorithms that can model high-level information in an image using several processing layers of transformations. This uses an architecture of multi-level linear and non-linear operations, described by layers, to learn complex functions that can represent high-level detail to map the input data to the output segmentations directly. As more data becomes available the algorithm can become more accurate and generalised, without a need to rewrite the underlying methods, thus making it a good choice for long term development. 

In recent years, deep learning-based methods have been applied to the segmentation of medical images, especially successful has been the U-Net \cite{ronneberger_u-net_2015}. This modified fully \ac{CNN} architecture uses a number of convolution, pooling and up-sampling layers to detect features in the input data at multiple resolutions. The convolution layers convolve a learnable kernel with the input data to generate spatial feature maps that are passed to subsequent layers in the network. By adjusting the kernels, the resulting feature maps can be optimised to detect the location of the kidneys. Pooling layers are used to down-sample the data and allow some convolution kernels to become tuned to approximate features, this also reduces the tendency of the network to overfit the training data. When the data has been fully down-sampled, up-sampling layers are used to increase the resolution of the feature maps back to that of the original data while more convolution layers also learn the precise location of the kidneys. Parameters are adjusted by comparing the output from the network to a known ground truth. \ac{CNN} methods have been applied to segmentation in other areas of medical imaging \cite{lu_automatic_2017, sharma_automatic_2017, wachinger_deepnat_2018, fu_novel_2018}, for example to prostate segmentation of \ac{MRI} images \cite{hassanzadeh_convolutional_2019}, liver segmentation of x-ray \ac{CT} images \cite{li_h-denseunet_2018} and segmentation of polycystic kidneys \cite{kline_performance_2017, van_gastel_automatic_2019, shin_expert-level_2020}. However, to date, these methods have not been successfully applied to \ac{CKD} and healthy kidney segmentation from MR images. 

Here a single 2D U-Net model \ac{CNN} is used for the segmentation of the kidneys in both \ac{HC} participants and \ac{CKD} patients using \ttwo-weighted MR images. Automatically generated kidney masks are compared with manual masks defined by experts and assessed for similarity using multiple voxel and surface based metrics and total segmented volume. A subset of subjects were scanned multiple times to assess the repeatability of the segmentations.

\newpage

\section{Neural Networks for Image Segmentation}

\subsection{Artificial Neural Networks}
\acp{ANN} aim to solve computational problems using a similar methodology to their biological namesake. Input data is passed through a series of connected nodes or neurons, each of which can have multiple input and output connections from and to other neurons mimicking synapses. At each neuron, a weighted sum of the input values is calculated before being passed onto the next hidden layer of neurons. The final layer of neurons is connected to the output layer which will give an estimation of the desired property, be that a number e.g. probability someone will like a television program, an image e.g. the probability that a pixel in an image is a road sign, or a sample in a time series e.g. audio in voice synthesis. More concisely, an \ac{ANN} can be used to map a non-linear set of input data to an output dimension.

A very basic example could use the mass and colour of an animal to guess if it is a dog or a cat, Figure \ref{fig:ml_theory_init}. The connections between neurons are initialised with random weights. 

\begin{figure}[H]
	\centering
	\includegraphics[width=0.8\textwidth]{ML/theory/ANN_init.eps}
	\caption{The \ac{ANN} initialised with random numbers trying to predict the species of a small ginger cat.}
	\label{fig:ml_theory_init}	
\end{figure}

At each neuron, a weighted sum of its inputs is taken, then an activation function applied, here a \ac{ReLU}, Figure \ref{fig:ml_theory_activation_relu}, for the hidden neurons and sigmoid, \ref{fig:ml_theory_activation_sigmoid}, for the output neurons. These activation functions allow the network to act non-linearly and are modelling the action potential of biological neurons. The \ac{ReLU} function represents a higher rate of firing for signals above zero; as it is impossible for a biological neuron to reduce its firing rate below zero, the \ac{ReLU} outputs zero when the input signal is negative. The sigmoid function maps all values between zero and one, and therefore ensures the network outputs a probability at the output nodes.

\begin{figure}[H]
	\centering
	\begin{subfigure}[c]{0.47\textwidth}
		\centering
		\includegraphics[width=1\textwidth]{ML/theory/relu.pdf}
		\caption{$\sigma (x) = max(0, x)$}
		\label{fig:ml_theory_activation_relu}
	\end{subfigure}
	\hfill
	\begin{subfigure}[c]{0.47\textwidth}
		\centering
		\includegraphics[width=1\textwidth]{ML/theory/sigmoid.pdf}
		\caption{$\sigma(x) = \frac{1}{1 + e^{-x}}$}
		\label{fig:ml_theory_activation_sigmoid}
	\end{subfigure}
	\caption{Activation functions.}
	\label{fig:ml_theory_activation}
\end{figure}

As the weights were randomly initialised, the network has incorrectly predicted that the animal is a dog. By comparing the result output from the network to the known ground truth, the weights of the network can be adjusted in a process known as back propagation, Figure \ref{fig:ml_theory_adjusted}. Hyper-parameters such as learning rate and momentum control how much each weight is adjusted in response to the input data and subsequent result.

\begin{figure}[H]
	\centering
	\includegraphics[width=0.8\textwidth]{ML/theory/ANN_back_prop.eps}
	\caption{The weightings of each connection are adjusted so the output layer produces results closer to the ground truth.}
	\label{fig:ml_theory_adjusted}	
\end{figure}

When another animal is input to the network, here a smaller, darker coloured cat, the network now correctly predicts that it is a cat, \ref{fig:ml_theory_new_data}. By repeating this procedure many times, comparing the result to ground truths and adjusting the weights, a process known as training, the network becomes more and more accurate. Once trained, the network can be used to infer the species of animals with no ground truth data.

\begin{figure}[H]
	\centering
	\includegraphics[width=0.8\textwidth]{ML/theory/ANN_new_data.eps}
	\caption{New data is presented to the network in the form of a smaller, darker coloured cat and the process of adjusting weights is repeated until more accurate results are produced.}
	\label{fig:ml_theory_new_data}	
\end{figure}

The above example is highly simplified, real \ac{ANN}s will have many more input nodes and hidden layers. In the case of imaging data, the input layer will simply be a node for each pixel in the image.

\subsection{Convolutional Neural Networks}

It was found that \ac{ANN} segmentation performance was increased if additional features were input to the network. These features could be colour rather than greyscale date, different \ac{MRI} contrasts e.g. fat/water images or artificially generated features. In Figure \ref{fig:ml_theory_features} a selection of artificially generated features are presented, some of these highlight the kidneys from the surrounding tissue e.g. the intensity range adjustment, \ref{fig:ml_theory_features_levels}, and edge detector, \ref{fig:ml_theory_features_edges} while others are better at providing contrast between the cyst in the right kidney and the renal tissue e.g. the intensity inversion, \ref{fig:ml_theory_features_invert} and sharpening filter, \ref{fig:ml_theory_features_sharpern}.

\begin{figure}[H]
	\centering
	\begin{subfigure}[c]{0.30\textwidth}
		\centering
		\includegraphics[width=1\textwidth]{ML/Theory/Features/raw.png}
		\caption{}
		\label{fig:ml_theory_features_raw}
	\end{subfigure}
	\hfill
	\begin{subfigure}[c]{0.30\textwidth}
		\centering
		\includegraphics[width=1\textwidth]{ML/Theory/Features/levels.png}
		\caption{}
		\label{fig:ml_theory_features_levels}
	\end{subfigure}
	\hfill
	\begin{subfigure}[c]{0.30\textwidth}
		\centering
		\includegraphics[width=1\textwidth]{ML/Theory/Features/invert.png}
		\caption{}
		\label{fig:ml_theory_features_invert}
	\end{subfigure}
	\vskip\baselineskip
	\begin{subfigure}[c]{0.30\textwidth}
		\centering
		\includegraphics[width=1\textwidth]{ML/Theory/Features/blur.png}
		\caption{}
		\label{fig:ml_theory_features_blur}
	\end{subfigure}
	\hfill
	\begin{subfigure}[c]{0.30\textwidth}
		\centering
		\includegraphics[width=1\textwidth]{ML/Theory/Features/sharpern.png}
		\caption{}
		\label{fig:ml_theory_features_sharpern}
	\end{subfigure}
	\hfill
	\begin{subfigure}[c]{0.30\textwidth}
		\centering
		\includegraphics[width=1\textwidth]{ML/Theory/Features/edges.png}
		\caption{}
		\label{fig:ml_theory_features_edges}
	\end{subfigure}
	\caption{An example of the features that can be generated from a raw image (\subref{fig:ml_theory_features_raw}). Implemented here are intensity range adjustments (\subref{fig:ml_theory_features_levels}), intensity inversion (\subref{fig:ml_theory_features_invert}), Gaussian blur (\subref{fig:ml_theory_features_blur}), sharpening (\subref{fig:ml_theory_features_sharpern}) and edge detection (\subref{fig:ml_theory_features_edges}).}
	\label{fig:ml_theory_features}
\end{figure}

Many of these artificially generated features can be implemented as convolutional operations, these involve convolving a numerical kernel with every pixel in the image. By adjusting the values of each cell in the kernel, different features or filters can be produced. The control of these kernels can be handed over to similar optimisation processes to those used to adjust the weights of the connections between neurons. Over the training period, this enables the network to learn what features are useful for the task at hand and which as less useful rather than the network being given features that the programmer thinks will be helpful. Shallow layers of the network usually resemble features similar to those in \ref{fig:ml_theory_features} while deeper layers represent more complex and specific objects such as, using the cat and dog example above, pointy noses to distinguish dogs and triangular ears to distinguish cats. This architecture is known as a \acl{CNN} \cite{fukushima_neocognitron_1988,lecun_gradient-based_1998}. 

To aid with feature extraction, the raw image is often downsampled by max-pooling layers, this enables different kernels to act on different scales of the image can help keep the network generalisable and avoid overfitting. The U-Net architecture \cite{ronneberger_u-net_2015} combines an arm with downsampling and feature extraction with an upsampling arm that returns the image to its initial dimensions making it especially useful for segmentations tasks.

\newpage
\section{Methods}
The study was approved by the University of Nottingham Medical School Research Ethics Committee (H14082014 and E14032013), and East Midlands Research Ethics committee REC reference: 17/LO/2036 and 15/EM/0274.

\subsection{MRI Data Acquisition}
\label{sec:ml_methods_acquisition}

All kidney \ac{MRI} scans were acquired on a 3T Philips Ingenia system (Philips Medical Systems, Best, The Netherlands) using a 2D \ttwo-weighted \ac{HASTE} sequence optimised to achieve the maximum contrast between the kidneys and surrounding tissue (\ac{TE} = 60 ms, \ac{TR} = 1300 – 1800 ms, \ac{SENSE} factor = 2.5, refocus angle 120$\degree$, bandwidth, 792 Hz, \ac{FOV} = 350 x 350 mm$^2$, voxel size = 1.5 x 1.5 x 5 mm$^3$ and a slice gap of 0.5 mm with approximately 13 coronal slices, enough to image the entire kidney \cite{petzold_building_2014, will_automated_2014}, in a single 17 - 23 s breath hold.

The dataset consisted of 60 subjects, 30 \ac{HC} (10 female, 20 male) with a mean age of 26 $\pm$ 11 (19–77) years and 30 \ac{CKD} patients (6 female, 24 male) with a mean age of 59 $\pm$ 14 (19–80) years and mean \ac{CKD} Stage 3.5 $\pm$ 1.2 (1-5). Ten of the subjects (5 \ac{HC}s and 5 \ac{CKD} patients) were scanned five times in the same scan session for use as test data. In each test data scan session, subjects were repositioned between each acquisition (removed from the scanner, asked to sit up and move on the bed), additionally the scanner operator attempted to vary the acquisition geometry between each scan while still acquiring full kidney coverage. These repeated test datasets allow the consistency of the networks ability to measure \ac{TKV} to be assessed. 

In total, 649 2D image slices from the 50 subjects in the training data and 650 2D image slices from the 10 subjects in the test data, were collected. A summary of the data collected is provided in Table \ref{tab:ml_datasets} and Figure \ref{fig:ml_true_tkv_hist}.

\subsection{Manual Segmentation}
The manual binary mask of the kidneys of each subject were generated by one of three observers (A, B and C who had been trained on kidney segmentation and had an average of 2 years of experience), with each observer segmenting data from both the training and testing datasets. Kidney boundaries were manually traced using freely available software (MRIcron \cite{rorden_neurolabuscmricron_2021}) and any area of non-renal parenchyma, such as the renal hilum and cysts, were excluded from the manual definition. Binary masks of the kidney were generated, and the volume of each kidney was computed from the product of the number of voxels in each kidney mask and the voxel volume. Separate kidney volume for the left and right kidneys was determined and summed to compute \ac{TKV}. All measurements were performed by observers blinded for patient number and previous \ac{TKV} measurements. 

For the training phase, for each subject a manual mask was used from a single observer (randomised between observer A, B, or C). For the testing phase, all five scans from a given subject were segmented by a single reader with the ten subjects being segmented by a mix of the three readers i.e. the test data comprised of subjects segmented by all readers but the repeat scans of each subject were segmented by the same reader. For four \ac{HC} subjects from the test dataset, manual masks were drawn by all three observers for all five repeat acquisitions to allow assessment of inter-observer variability in the manual masks. \ac{HC}s were chosen for this analysis as they healthy kidneys have a more consistent morphology and thus will give a best-case measure of observer variability and provide a comparison of the automated method to the highest standard of manual segmentation.

\subsection{Automated Segmentation Using a CNN Architecture}

Voxel intensities were normalised between 0 and 255, where 0 was set to the mean voxel intensity minus 0.5 times the standard deviation of that slice and 255 was set to the mean voxel intensity plus four times the standard deviation of the volume. This empirically derived windowing led to a clear contrast between the kidneys and surrounding tissue while negating the effects of bulk signal changes between volumes. Each dataset volume was then split into 2D coronal slices and resampled to a matrix size of 256 $\times$ 256. Twenty percent of slices were reserved for validation during the network optimisation process, this validation data was used to monitor over-fitting and direct the optimisation process between epochs. Once the data had been split into training and validation sets, the slice order was randomised within sets. Splitting the data before slice randomisation limited the possibility of slices from only one  subject being split over both the training and validation datasets. During training, data augmentation was applied. At the start of each epoch, a batch of images and their corresponding masks was selected at random from the training data and a series of random shifts (up to 25 \% of the image in both the horizontal and vertical direction), zooms (between 0.75 and 1.25 magnification), rotations (within a 20$\degree$ range), and sheers (within a 5$\degree$ range) were applied to the image/mask pair to produce different yet anatomically reasonable images. The weights of the network were then adjusted based on this augmented data before selecting a new batch of images for the next epoch. Augmenting the data reduces the tendency of a model to over-fit the training data and thus increases accuracy when the model is applied to unseen images. 

The U-Net consists of two Fully Convolutional Neural Network-like structures that are cascaded in the form of an encoder-decoder (autoencoder) structure. The encoder is used for feature extraction and the decoder is used for feature mapping to the original input resolution. A summary of the network architecture is shown in Figure \ref{fig:ml_network}. The convolution layers use a set of small parameterised filters, referred to as kernels, to perform convolution operations to produce different feature maps of their input. Here each convolution and deconvolution layer uses a 3 x 3 kernel. Activation layers use a \ac{ReLU}. Following convolution at each resolution, max pooling with a stride 2 is used on the encoding half of the network.

\begin{figure}[h]
	\centering
	\includegraphics[width=1\textwidth]{ML/Model/Model.eps}
	\caption{The architecture of the network used.}
	\label{fig:ml_network}	
\end{figure}

The network was implemented using Keras (v2.2.4) \cite{chollet_keras_2015} with a TensorFlow backend (v1.13.1) \cite{abadi_tensorflow_2015} in Python 3.6.9. All training was carried out on an NVIDIA Titan Xp \ac{GPU} (3840 CUDA cores, 12 GB GDDR5X). The network uses a Dice score loss function, given by,
\begin{equation}
	D\left(A, B\right) = \frac{2\left| A \cap B \right|}{\left|A\right|+\left|B\right|} = \frac{2TP}{2TP + FP + FN},
	\label{eq:dice}
\end{equation} 
where $TP$ is true positive, $FP$ is false positive and $FN$ is false negative. A value of 1 implies complete overlap between the automated mask and the manual mask while 0 implies no overlap. This function is ideal for renal segmentation as it does not weight true negatives which represent the majority of voxels input to the network and thus means that while the network is training, it does not become trapped in a local minimum outputting solely background voxels. Training was carried out over 150 epochs using stochastic gradient descent with an initial learning rate of 0.01 and learning rate decay of $5\times10^{-7}$ and momentum of 0.8, these parameters help the optimiser converge quickly while also avoiding overshooting. As seen in Figure \ref{fig:ml_training_history}, after 150 epochs the validation Dice score plateaued while the training Dice score was still rising slightly, indicating that any further training would lead to over-fitting. Training took approximately thirty minutes.

\begin{figure}[h]
	\centering
	\includegraphics[width=0.7\textwidth]{ML/Training_progress/training_history.pdf}
	\caption{Dice score of the network for the training and validation data. Data is shown with a 10 epoch rolling average.}
	\label{fig:ml_training_history}	
\end{figure}

\newpage
\subsection{Statistical Analysis}

Baseline demographics are reported as mean $\pm$ \ac{SD}. Inter-observer variability in manual segmentation and \ac{TKV} was calculated by comparing the \ac{TKV} of the manual masks each observer generated for a given volume, and also assessing the Bland-Altman and regression analysis. Intra-observer variability in manual segmentation was calculated by comparing the \ac{TKV} of the five masks generated by an observer for a given subject. For each, the mean \ac{CoV}; defined as standard deviation/mean and \ac{ICC} were used as measures of repeatability of \ac{TKV}. Voxel-based metrics (Dice score, Equation \eqref{eq:dice} and Jaccard index, Equation \eqref{eq:jaccard}) and surface based metrics such as the average distance between the surface of the two masks and Hausdorff Distance 95th percentile, the 95th percentile of the largest distance between the two surfaces, were also calculated between each observer.

\begin{equation}
	J(A,B) = \frac{A\cap B}{A\cup B} = \frac{TP}{TP+FP+FN}
	\label{eq:jaccard}
\end{equation}
	
The performance of the automated segmentation was assessed using the voxel and surface based similarity metrics outlined above and, in addition, sensitivity, specificity, precision and accuracy, Equations \eqref{eq:sensitivity} - \eqref{eq:accuracy}. Performance was further assessed by determining the mean difference in \ac{TKV} between the automatic and manual methods. Both actual and percentage (\%) difference in \ac{TKV} were evaluated. Bias (mean) obtained from the automatic and manual methods was assessed using a paired sample t-test. The mean \ac{CoV} and \ac{ICC} were also used as measures of repeatability of the automated \ac{TKV}.

\begin{eqnarray}
	\textup{Sensitivity} &=& \frac{TP}{TP + FN}
	\label{eq:sensitivity}\\
	\textup{Specificity} &=& \frac{TN}{TN + FP}
	\label{eq:specificity}\\
	\textup{Precision} &=& \frac{TP}{TP + FP}
	\label{eq:precision}\\
	\textup{Accuracy} &=& \frac{TP + TN}{TP + TN + FP + FN}
	\label{eq:accuracy}
\end{eqnarray}

\newpage
\section{Results}

\subsection{Characteristics of the Training Cohort}

Data was collected using a \ttwo-weighted \ac{HASTE} sequence providing optimal contrast between the kidneys and surrounding tissue, examples shown in Figure \ref{fig:ml:raw}, however there is limited contrast between the left kidney and spleen due to their similar \ttwo-weighting. Cysts of variable size are clearly visible in the kidneys of the \ac{CKD} patient. 

\begin{figure}[H]
	\centering
	\begin{subfigure}[c]{0.47\textwidth}
		\centering
		\includegraphics[width=1\textwidth]{ML/Raw_data/HC_Raw.eps}
		\caption{}
		\label{fig:ml_raw_hc}
	\end{subfigure}
	\hfill
	\begin{subfigure}[c]{0.47\textwidth}
		\centering
		\includegraphics[width=1\textwidth]{ML/Raw_data/CKD_Raw.eps}
		\caption{}
		\label{fig:ml_raw_ckd}
	\end{subfigure}
	\caption{All slices of the raw data from representative subjects of the \ac{HC} cohort, shown in \subref{fig:ml_raw_hc}, and \ac{CKD} cohort, \subref{fig:ml_raw_ckd}.}
	\label{fig:ml_raw}
\end{figure}

The training data comprised 25 healthy controls (9 female, 16 male) with a mean age of 26 $\pm$ 12 (19–77) years and 25 \ac{CKD} patients (6 female, 19 male) with a mean age of 58 $\pm$ 15 (19–80) years and mean \ac{CKD} stage 3.3 $\pm$ 1.1 (1-5). The manual \ac{TKV} was 277 $\pm$ 60 m$\ell$, ranging between 145 and 422 m$\ell$. Including both healthy control subjects and \ac{CKD} patients meant the kidneys had variable morphology (shape, size and heterogeneous cysts) within the training dataset. Table \ref{tab:ml_datasets} provides the characteristics of datasets used for training and testing of the \ac{CNN}, whilst Figure \ref{fig:ml_true_tkv_hist} shows the distribution of \ac{TKV} within the training and testing data.
\begin{table}[H]
	\centering
	\begin{adjustbox}{width=1.1\textwidth, center}
	\begin{tabularx}{1.1\textwidth}{X|X|X|X|X|X|X}
		& \textbf{Number of Subjects} & \textbf{Number of Datasets} & \textbf{Number of 2D Slices} & \textbf{Sex (F/M)} & \textbf{Mean Age} & \textbf{TKV (m$\ell$)} \\ \hline
		\textbf{Training HC}  & 25                          & 25                          & 325                          & 9/16               & 26 $\pm$ 12       & 296 $\pm$ 38           \\
		\hline
		\textbf{Training CKD} & 25                          & 25                          & 324                          & 6/19               & 58 $\pm$ 15       & 258 $\pm$ 72           \\
		\hline
		\textbf{Testing HC}   & 5                           & 25                          & 325                          & 1/4                & 25 $\pm$ 3        & 330 $\pm$ 35           \\
		\hline
		\textbf{Training CKD} & 5                           & 25                          & 325                          & 0/5                & 69 $\pm$ 3        & 274 $\pm$ 56          
	\end{tabularx}
	\end{adjustbox}
	\caption{Characteristics of datasets used for training and validation of the 2D U-Net model \ac{CNN}.}
	\label{tab:ml_datasets}
\end{table}
\begin{figure}[H]
	\centering
	\includegraphics[width=0.7\textwidth]{ML/Datasets/dataset_hist_overlay.pdf}
	\caption{Distribution of \ac{TKV} within the training and testing data.}
	\label{fig:ml_true_tkv_hist}	
\end{figure}

\subsection{Reducing Acquisition Time}
Initial data was collected with a \ac{TR} of 1800 ms leading to a breath hold of approximately 23 seconds. Some subjects struggled to hold their breath for this long on expiration, therefore the effects of reducing the \ac{TR} of the sequence were investigated. As can be seen in Figure \ref{fig:ml_tr}, there is no degradation in image quality from the image with \ac{TR} of 1800 ms to that with at \ac{TR} of 1300 ms, the differences between these images are mainly due to the small movements between volumes, as can be seen in the difference data where the largest differences are seen around the periphery of the kidneys and in the gut. Moving forward, the \ac{TR} was reduced to 1300 ms leading to a sequence with a breath hold of approximately 17 seconds.

\begin{figure}[H]
	\centering
	\includegraphics[width=1\textwidth]{ML/TR/TR_Master_V0_1.eps}
	\caption{The effects of changing the \ac{TR} of the sequence.}
	\label{fig:ml_tr}	
\end{figure}

\subsection{Accuracy of Manual Segmentation}
Four of the test subjects were each scanned five times, with the left and right kidneys in the 20 datasets each masked by Observers A, B and C. The intra-observer and inter-observer variability for this manual segmentation was computed, as shown in Table \ref{tab:ml_manual_repeatability}. 

\begin{table}[H]
	\centering
	\begin{tabular}{ll|l|l}
		Observer                 & Kidneys & CoV (\%)      & ICC   \\ \hline
		\multirow{3}{*}{Intra A} & Total   & 2.2 $\pm$ 0.7 & 0.939 \\ \cline{2-4} 
		& Left    & 3.2 $\pm$ 0.8 & 0.783 \\ \cline{2-4} 
		& Right   & 1.9 $\pm$ 0.5 & 0.957 \\ \hline
		\multirow{3}{*}{Intra B} & Total   & 1.9 $\pm$ 0.3 & 0.895 \\ \cline{2-4} 
		& Left    & 2.0 $\pm$ 0.5 & 0.807 \\ \cline{2-4} 
		& Right   & 2.4 $\pm$ 0.3 & 0.892 \\ \hline
		\multirow{3}{*}{Intra C} & Total   & 2.5 $\pm$ 0.9 & 0.908 \\ \cline{2-4} 
		& Left    & 2.8 $\pm$ 1.3 & 0.769 \\ \cline{2-4} 
		& Right   & 3.1 $\pm$ 1.9 & 0.940 \\ \hline
		\multirow{3}{*}{Inter}   & Total   & 3.0 $\pm$ 1.0 & 0.897 \\ \cline{2-4} 
		& Left    & 4.0 $\pm$ 1.4 & 0.713 \\ \cline{2-4} 
		& Right   & 2.9 $\pm$ 1.0 & 0.910
	\end{tabular}
	\caption{Repeatability of the manual segmentation for left, right and \ac{TKV}, with coefficient of variation and intraclass coefficient computed.}
	\label{tab:ml_manual_repeatability}
\end{table}

Additionally, similarity metrics were used to assess the overlap between each observer’s manual masks, Table \ref{tab:ml_manual_metrics}. Due to the large difference between in-plane and out-of-plane resolution (1.5 mm$^3$ vs 5.5 mm$^3$) the Hausdorff distance is very susceptible to inaccuracies in the anterior – posterior direction; this metric is highly sensitive to noise and as such the 95th percentile is used to generate a more representative value. Bland-Altman plots and regression analysis of inter-observer variance in measured \ac{TKV} are provided in Figure \ref{fig:ml_obs_var}.

\begin{table}[H]
	\centering
	\begin{adjustbox}{width=1.2\textwidth, center}
	\begin{tabularx}{1.25\textwidth}{XX|X|X|X|X|X}
		Observer               & Kidney & Dice   Score  & Jaccard   Index & Average   Distance (mm) & Hausdorff   Distance (mm) (95th Percentile) & Volume   Difference (ml) \\ \hline
		\multirow{3}{*}{A – B} & Both   & 0.93   $\pm$ 0.03 & 0.87   $\pm$ 0.05   & 0.81   $\pm$ 0.58           & 5.59   $\pm$ 2.77                               & 20.84   $\pm$ 9.33           \\ \cline{2-7} 
		& Left   & 0.92 $\pm$ 0.07   & 0.85 $\pm$ 0.10     & 0.94 $\pm$ 1.12             & 5.53 $\pm$ 3.65                                 & 13.36 $\pm$ 5.76             \\ \cline{2-7} 
		& Right  & 0.94   $\pm$ 0.01 & 0.88   $\pm$ 0.02   & 0.65   $\pm$ 0.14           & 4.75   $\pm$ 1.15                               & 7.48   $\pm$ 5.63            \\ \hline
		\multirow{3}{*}{A – C} & Both   & 0.93 $\pm$ 0.01   & 0.87 $\pm$ 0.02     & 0.79 $\pm$ 0.18             & 5.83 $\pm$ 1.86                                 & 16.01 $\pm$ 8.56             \\ \cline{2-7} 
		& Left   & 0.93   $\pm$ 0.01 & 0.87   $\pm$ 0.02   & 0.84   $\pm$ 0.27           & 6.83   $\pm$ 3.12                               & 6.93   $\pm$ 5.78            \\ \cline{2-7} 
		& Right  & 0.93 $\pm$ 0.01   & 0.87 $\pm$ 0.02     & 0.72 $\pm$ 0.17             & 4.82 $\pm$ 1.25                                 & 9.08 $\pm$ 5.41              \\ \hline
		\multirow{3}{*}{B – C} & Both   & 0.94   $\pm$ 0.04 & 0.89   $\pm$ 0.06   & 0.63   $\pm$ 0.62           & 3.59   $\pm$ 2.74                               & -4.83   $\pm$ 9.92           \\ \cline{2-7} 
		& Left   & 0.93 $\pm$ 0.08   & 0.88 $\pm$ 0.11     & 0.78 $\pm$ 1.22             & 4.31 $\pm$ 3.58                                 & -6.44 $\pm$ 6.17             \\ \cline{2-7} 
		& Right  & 0.95   $\pm$ 0.01 & 0.90   $\pm$ 0.02   & 0.48   $\pm$ 0.14           & 3.39   $\pm$ 1.15                               & 1.61   $\pm$ 6.56           
	\end{tabularx}
	\end{adjustbox}
	\caption{Metrics comparing each combination of observers manual masks (A – B, A – C and B – C). All values are quoted as mean $\pm$ standard deviation.}
\label{tab:ml_manual_metrics}
\end{table}

\begin{figure}[H]
	\centering
	\begin{subfigure}[c]{0.32\textwidth}
		\centering
		\includegraphics[width=1\textwidth]{ML/Observer_variability/Corr_A_B.pdf}
		\caption{}
		\label{fig:ml_obs_var_cor_ab}
	\end{subfigure}
	\hfill
	\begin{subfigure}[c]{0.32\textwidth}
		\centering
		\includegraphics[width=1\textwidth]{ML/Observer_variability/Corr_A_C.pdf}
		\caption{}
		\label{fig:ml_obs_var_cor_ac}
	\end{subfigure}
	\hfill
	\begin{subfigure}[c]{0.32\textwidth}
		\centering
		\includegraphics[width=1\textwidth]{ML/Observer_variability/Corr_B_C.pdf}
		\caption{}
		\label{fig:ml_obs_var_cor_bc}
	\end{subfigure}
	\vskip\baselineskip
	\begin{subfigure}[c]{0.32\textwidth}
		\centering
		\includegraphics[width=1\textwidth]{ML/Observer_variability/ab_ba.pdf}
		\caption{}
		\label{fig:ml_obs_var_ba_ab}
	\end{subfigure}
	\hfill
	\begin{subfigure}[c]{0.32\textwidth}
		\centering
		\includegraphics[width=1\textwidth]{ML/Observer_variability/ac_ba.pdf}
		\caption{}
		\label{fig:ml_obs_var_ba_ac}
	\end{subfigure}
	\hfill
	\begin{subfigure}[c]{0.32\textwidth}
		\centering
		\includegraphics[width=1\textwidth]{ML/Observer_variability/bc_ba.pdf}
		\caption{}
		\label{fig:ml_obs_var_ba_bc}
	\end{subfigure}
	\caption{}
	\label{fig:ml_obs_var}
\end{figure}
\subsection{Network Testing}
To verify that the trained network is behaving as expected saliency maps were produced, Figure \ref{fig:ml_salency}, this is especially important given the black box nature of machine learning methods. This map shows the areas the network is using most in its classification \cite{mahapatra_visual_2016}. It verifies that the networks is using the outside areas of the kidney to make its prediction with areas of a similar intensity receiving some attention to distinguish them from the kidney. While this is precisely what is expected of the algorithm, it is important to check this as it is possible for such a method to have learnt a slightly different mechanism for the segmentation, one that is more prone to errors if new data is presented to it.

\begin{figure}[H]
	\centering
	\includegraphics[width=.4\textwidth]{ML/Salency/MSE_Salency.png}
	\caption{An example saliency map of the areas the network uses most when segmenting the kidney.}
	\label{fig:ml_salency}	
\end{figure}

The trained network was used to predict segmentations of the 2D kidney slices and compute \ac{TKV} for each of the unseen test volumes. The mean Dice score over the 50 test volumes was 0.93 $\pm$ 0.01 (0.94 $\pm$ 0.02 for HC and 0.92 $\pm$ 0.01 for \ac{CKD} patients). The \ac{TKV} predicted by the network was, on average, 1.2 $\pm$ 16.2 m$\ell$ less than the manually segmented \ac{TKV} and thus not significantly different (p = 0.615) (Figure \ref{fig:ml_testing_ba}) This accuracy was comparable for the \ac{HC} and \ac{CKD} cohorts, with automated CNN \ac{TKV} measurements of 4.7 $\pm$ 17.7 m$\ell$ greater than manual and 7.0 $\pm$ 12.4 m$\ell$ less than manual respectively. A summary of the \ac{CNN} accuracy when evaluated using similarity metrics and volume difference from manual measures is shown in Table \ref{tab:ml_testing_metrics}. Note a slightly larger discrepancy for the left compared to the right kidney. Figure \ref{fig:ml_testing_ba} shows plots of the difference in volume between manual segmentation and automated segmentation of the test dataset. 

\begin{figure}[H]
	\centering
	\begin{subfigure}[c]{0.47\textwidth}
		\centering
		\includegraphics[width=1\textwidth]{ML/BA_plots/rescale_whole_image_max_dice_09177_validation_ba_volume.pdf}
		\caption{}
		\label{fig:ml_testing_ba_volume}
	\end{subfigure}
	\hfill
	\begin{subfigure}[c]{0.47\textwidth}
		\centering
		\includegraphics[width=1\textwidth]{ML/BA_plots/rescale_whole_image_max_dice_09177_validation_ba_percent.pdf}
		\caption{}
		\label{fig:ml_testing_ba_percent}
	\end{subfigure}
	\caption{The difference between the \ac{TKV} predicted by the \ac{CNN} and the manually segmented true \ac{TKV}. Mean and standard deviation \ac{TKV} difference are shown as dashed and dotted lines respectively. Each subject is shown in a different colour. (\subref{fig:ml_testing_ba_volume}) shows the absolute volume difference (\subref{fig:ml_testing_ba_percent}) shows the percentage volume difference.}
	\label{fig:ml_testing_ba}
\end{figure}

\begin{table}[H]
	\centering
	\begin{adjustbox}{width=1.2\textwidth, center}
	\begin{tabularx}{1.5\textwidth}{XX|X|X|X|X|X|X|X|X|X}
		Cohort               & Kidney & Dice   Score  & Jaccard   Index & Sensitivity   & Specificity     & Precision     & Accuracy        & Mean   Surface Distance (mm) & Hausdorff   Distance (mm) (95th Percentile) & Volume   Difference (p)  \\ \hline
		\multirow{3}{*}{All} & Total  & 0.93 $\pm$   0.01 & 0.87 $\pm$   0.03   & 0.93 $\pm$   0.03 & 0.997   $\pm$ 0.001 & 0.93 $\pm$   0.02 & 0.995   $\pm$ 0.001 & 0.65 $\pm$   0.21                & 4.33 $\pm$   1.64                               & -1.16   $\pm$ 16.23 (0.615)  \\ \cline{2-11} 
		& Left   & 0.92 $\pm$ 0.02   & 0.86 $\pm$ 0.04     & 0.91 $\pm$ 0.05   & 0.997 $\pm$ 0.001   & 0.94 $\pm$ 0.03   & 0.994 $\pm$ 0.002   & 0.76 $\pm$ 0.31                  & 4.42 $\pm$ 1.52                                 & -3.95 $\pm$ 12.38   (0.029)  \\ \cline{2-11} 
		& Right  & 0.94 $\pm$   0.02 & 0.89 $\pm$   0.03   & 0.95 $\pm$   0.03 & 0.997   $\pm$ 0.001 & 0.93 $\pm$   0.03 & 0.996   $\pm$ 0.001 & 0.54 $\pm$   0.21                & 3.66 $\pm$   1.76                               & 2.79 $\pm$   6.84 (0.006)    \\ \hline
		\multirow{3}{*}{HC}  & Total  & 0.94 $\pm$ 0.02   & 0.88 $\pm$ 0.03     & 0.95 $\pm$ 0.05   & 0.997 $\pm$ 0.001   & 0.93 $\pm$ 0.03   & 0.995 $\pm$ 0.001   & 0.68 $\pm$ 0.27                  & 4.50 $\pm$ 1.97                                 & 4.66 $\pm$ 17.72 (0.201)     \\ \cline{2-11} 
		& Left   & 0.93 $\pm$   0.02 & 0.87 $\pm$   0.04   & 0.94 $\pm$   0.05 & 0.997   $\pm$ 0.001 & 0.93 $\pm$   0.03 & 0.994   $\pm$ 0.002 & 0.79 $\pm$   0.37                & 4.47 $\pm$   1.81                               & 1.91 $\pm$   12.93 (0.467)   \\ \cline{2-11} 
		& Right  & 0.95 $\pm$ 0.02   & 0.90 $\pm$ 0.03     & 0.96 $\pm$ 0.03   & 0.997 $\pm$ 0.001   & 0.94 $\pm$ 0.02   & 0.996 $\pm$ 0.001   & 0.56 $\pm$ 0.26                  & 3.81 $\pm$ 2.11                                 & 2.75 $\pm$ 7.70   (0.087)    \\ \hline
		\multirow{3}{*}{CKD} & Total  & 0.92 $\pm$   0.01 & 0.86 $\pm$   0.02   & 0.91 $\pm$   0.02 & 0.998   $\pm$ 0.001 & 0.94 $\pm$   0.02 & 0.995   $\pm$ 0.001 & 0.63 $\pm$   0.14                & 4.16 $\pm$   1.24                               & -6.98   $\pm$12.38 (0.009)   \\ \cline{2-11} 
		& Left   & 0.92 $\pm$ 0.02   & 0.85 $\pm$ 0.03     & 0.89 $\pm$ 0.04   & 0.998 $\pm$ 0.001   & 0.95 $\pm$ 0.02   & 0.994 $\pm$ 0.002   & 0.73 $\pm$ 0.24                  & 4.37 $\pm$ 1.21                                 & -9.81 $\pm$ 8.62   (0.00001) \\ \cline{2-11} 
		& Right  & 0.93 $\pm$   0.01 & 0.88 $\pm$   0.02   & 0.94 $\pm$   0.02 & 0.997   $\pm$ 0.001 & 0.92 $\pm$   0.03 & 0.996   $\pm$ 0.001 & 0.51 $\pm$   0.13                & 3.51 $\pm$   1.34                               & 2.83 $\pm$   6.02 (0.027)   
	\end{tabularx}
	\end{adjustbox}
	\caption{The accuracy of the \ac{CNN} compared to manual segmentations using a variety of metrics stratifying the testing data by cohort and left vs right kidney. All values are given as mean $\pm$ standard deviation. }
	\label{tab:ml_testing_metrics}
\end{table}

In Figure \ref{fig:ml_testing_corr}, the \ac{TKV} predicted by the CNN is plot against the manual \ac{TKV}, in ninety percent of subjects, the standard deviation of \ac{TKV} measurements between each volume for a subject was smaller when the \ac{TKV} was measured using the \ac{CNN} as oppose to manually. The mean \ac{CoV} and \ac{ICC} were 2.7 $\pm$ 0.9 \% and 0.979 respectively across the 5 repeats of the manually segmented test data (using masks from observers A, B and C), compared to a value of 1.5 $\pm$ 0.5 \% and 0.993 respectively for the automatic segmentations of the 5 repeats of test data. The \ac{CNN} produced a significantly lower \ac{CoV} than the manual segmentations (p = 0.008).

\begin{figure}[H]
	\centering
	\includegraphics[width=0.8\textwidth]{ML/BA_plots/rescale_whole_image_max_dice_09177_validation_tkv.pdf}
	\caption{The \ac{TKV} predicted by the \ac{CNN} plot against the manually segmented true \ac{TKV} with each subject plot in a different colour. The standard deviation measured using both methods is shown as error bars originating from the mean of each subject. The dotted line represents perfect correlation between the \ac{CNN} and manual segmentation.}
	\label{fig:ml_testing_corr}	
\end{figure}

Representative examples of the output from the network for both \ac{HC} and \ac{CKD} data are shown in Figure \ref{fig:ml_masks}. The automated \ac{CNN} accurately segments the kidneys, and for \ac{CKD} patients, often omits cysts from the masks. 

\begin{figure}[H]
	\centering
	\begin{subfigure}[c]{1.0\textwidth}
		\centering
		\includegraphics[width=0.9\textwidth]{ML/ROI/HC.eps}
%\missingfigure{HC Masks}
		\caption{}
		\label{fig:ml_masks_hc}
	\end{subfigure}
	\vskip\baselineskip
	\begin{subfigure}[c]{1.0\textwidth}
		\centering
		\includegraphics[width=0.9\textwidth]{ML/ROI/CKD.eps}
%\missingfigure{CKD Masks}
		\caption{}
		\label{fig:ml_masks_ckd}
	\end{subfigure}
	\caption{Representative raw test data and corresponding masks of a \ac{HC}, (\subref{fig:ml_masks_hc}), and \ac{CKD} subject, (\subref{fig:ml_masks_ckd}). Manually generated masks are shown in blue, automatically generated masks are shown in red and the overlap of the two is shown in magenta.}
	\label{fig:ml_masks}
\end{figure}

Since this is a 2D \ac{CNN}, it is important to assess the accuracy across the anterior – posterior 2D slices of the kidney. This was achieved by comparing the Dice score of the \ac{CNN} to the inter-reader Dice scores, Figure \ref{fig:ml_dice_slice}. A decrease in accuracy in the outer slices can be seen in both the \ac{CNN} and manual masks.

\begin{figure}[H]
	\centering
	\includegraphics[width=0.7\textwidth]{ML/dice_slice.pdf}
	\caption{Mean Dice score for 2D slices from anterior to posterior. The shaded area represents one standard deviation from the mean Dice score.}
	\label{fig:ml_dice_slice}	
\end{figure}

This decrease in accuracy manifests itself on the outer slices of the volume, where the proportion of kidney per slice is smaller and as such the 2D network, with a lack of spatial context in the anterior – posterior direction, finds these outer slices more challenging. This decrease in accuracy can partly be explained by the fact that larger structures (in terms of number of voxels) will in general produce higher scores for comparable errors because the vast majority of errors are on the perimeter of the kidney in each slice, slices with fewer voxels of kidney have a smaller area to perimeter ratio, Figure \ref{fig:ml_morphology}.

\begin{figure}[H]
	\centering
	\includegraphics[width=0.8\textwidth]{ML/Morphology/morphology_hex.pdf}
	\caption{A 2D histogram of the perimeter and area of each slice for both the \ac{HC} and \ac{CKD} cohorts.}
	\label{fig:ml_morphology}	
\end{figure}

\newpage

\section{Discussion}
In this chapter, a 2D \ac{CNN} has been trained to generate automatic segmentations of healthy control and \ac{CKD} patients. Segmentations of the left and right kidneys are computed from which total kidney volume is estimated. The \ac{CNN} was trained on both healthy control and \ac{CKD} kidneys with a range of \ac{TKV} (144.76 – 422.49 m$\ell$) which included the presence of cysts. The automated segmentation by the \ac{CNN} yielded a mean Dice score of 0.93 $\pm$ 0.01 and took an average time of 9 s to measure \ac{TKV} compared to 15 – 30 minutes \cite{zollner_assessment_2012} for manual segmentation. The automated \ac{CNN} can be run as a self-contained \ac{GUI} with the data and programme freely available \cite{daniel_alexdaniel654/renal_segmentor_2020} and thus avoid the need for complicated software setup. Note the software released at present can only be used to process coronal HASTE images and will not be accurate with other geometries/contrasts, for this, training of the network with a different dataset would be required and thus necessitate the use of a \ac{GPU}.

\begin{figure}[H]
	\centering
	\includegraphics[width=0.7\textwidth]{ML/gui.png}
	\caption{The \acl{GUI} used to segment kidneys.}
	\label{fig:ml_gui}	
\end{figure}

\newpage
\subsection{Evaluation of Methodology}

The network performed with high precision on the test data with a 1.2 $\pm$ 16.2 m$\ell$, statistically insignificant, discrepancy between manual and automated \ac{TKV} measurements. Table \ref{tab:ml_testing_metrics} shows the agreement between the \ac{CNN} and manual masks is higher for the right than left kidney, this is in part due to the proximity and lack of contrast between the left kidney and the spleen making distinguishing this boundary difficult for the \ac{CNN}. This difficulty also leads to inconsistencies in manual masks, borne out by the increased \ac{CoV} and decreased \ac{ICC} and similarity metrics of the left kidney when compared to the right kidney in Table \ref{tab:ml_manual_repeatability} and Table \ref{tab:ml_testing_metrics} assessing the variability in manual masks between observers. From Table \ref{tab:ml_testing_metrics} it can also be seen that the agreement between the \ac{CNN} and manual masks is greater for the healthy control cohort than the \ac{CKD} cohort, this is expected due to the increased variation in kidney morphology and the presence of cysts in the \ac{CKD} cohort. Figure \ref{fig:ml_testing_ba} shows that the difference between the manual \ac{TKV} and \ac{CNN} predicted \ac{TKV} is not dependent on the true \ac{TKV}, therefore the training data is balanced and well augmented as the network is able to accurately perform over the full range of kidney size in the test data. 

Here, five volumes of test data were collected for each subject by repositioning the subject in the scanner within an hour scan session, and therefore any variance in measured \ac{TKV} is purely due to inaccuracies in the kidney \ac{ROI} definition. On assessing the correlation between manual and \ac{CNN} measured \ac{TKV} in Figure \ref{fig:ml_testing_corr}, it can be seen that, in 90 \% of subjects the intra-observer variance in manual \ac{TKV} between the segmentation of the five volumes collected in each subject is larger than using the \ac{CNN} to estimate \ac{TKV}, as reflected by the lower \ac{CoV} and increased \ac{ICC} of the \ac{TKV} measured using the \ac{CNN} (\ac{CoV} 1.5 $\pm$ 0.5 \%, \ac{ICC} 0.993) compared to the manual measures (\ac{CoV} 2.7 $\pm$ 0.9 \%, ICC 0.979). As the network is trained on the kidney segmentations from three observers (A, B and C) , it has been optimised by inheriting the most accurate tendencies of each observer e.g. one observer may have been very accurate when excluding cysts but not as accurate at defining the kidney-spleen boundary. The network will have learnt to exclude cysts from this observer but to delineate between kidney and spleen from another observer. Thus the network can become more precise than each individual observers manual segmentations. This increased precision can be seen in Figure \ref{fig:ml_testing_ba} when compared to Figure \ref{fig:ml_testing_corr} where the variance in difference in \ac{TKV} is driven by the larger variance in manual \ac{TKV}. The smallest \ac{TKV} per subject is consistently overestimated when compared to its manual mask and vice versa the largest manual \ac{TKV} per subject is often an underestimation compared to the manual \ac{TKV}. 

Figure \ref{fig:ml_masks} illustrates the masks produced by the manual segmentation and the \ac{CNN} for both a \ac{HC} and \ac{CKD} patient. For the \ac{HC}, the \ac{CNN} includes more voxels around the edge of its mask than manual segmentation, and the network is more anatomically accurate e.g. where the interface between the kidney and spleen is very narrow, the \ac{CNN} predicts the kidney is adjacent to the spleen whilst the observers manual segmentation leaves a gap. The \ac{CKD} data shown in Figure \ref{fig:ml_masks_ckd} includes a cyst in each of the kidneys. The network was trained on a combination of healthy and \ac{CKD} data, with 19 of the 25 \ac{CKD} training datasets containing at least one cyst. The \ac{CNN} can be seen to segment out the cysts, despite their highly variable morphology and prevalence in the overall training data.
 
The amount of augmentation applied to the training data was empirically derived (random shifts up to 25 \% of the image in both the horizontal and vertical direction, zooms between 0.75 and 1.25 times magnification, rotations within a 20 degree range, and sheers within a 5 degree range) and led to the potential for large transforms being applied to the data and masks if the extremes of each transform were randomly selected. This large degree of augmentation was advantageous as it mirrors the large variation in acquisition planning in abdominal imaging. 

A 2D \ac{CNN} was used to process each 2D slice of a full volume, rather than a 3D volume. This was advantageous for the relatively small training dataset the network was optimised on, as it avoids overfitting and allows the network to easily be used on volumes of variable slice number. However, this can come at the expense of accuracy as 2D \ac{CNN}s do not leverage the information from adjacent slices in the segmentation as is done in 3D \ac{CNN}, but 3D \ac{CNN} come with a computational cost as a result of the increased number of parameters used. 3D networks have successfully been implemented on neural data using patching methods where the image volume is divided up into smaller cubes \cite{wachinger_deepnat_2018} to reduce memory requirements and allow for differing input shapes. While this works well in the brain, there are a number of reasons why this method may not be as successful for body applications. The out-of-plane resolution is significantly less than the in-plane resolution; this results in far fewer slices in one direction than the other two. To avoid overfitting for a certain number of slices e.g. training on a 11 slice image with a 113 patch, and subsequently the network not performing well when the patch is applied to a 16 slice image, the patch would need to be much smaller than the number of slices, thus diminishing the benefits of the 3D methodology. Additionally, the extra memory requirements for a 3D network limit the ease of use of the software for inference on many standard office computers.

\subsection{Future Directions}

Future work will explore alternative network architectures. As the main source of inaccuracy with the current network is its lack of slice-to-slice context, there are multiple architectures that can address this. A relatively simple method would be exploiting the fact that the Keras framework is designed to work with colour images. By combining three slices into a single image where each slice represents either the red, green or blue channel, Figure \ref{fig:ml_rgb_slices}, the network would be able to use contextual information about the adjacent slices to help inform its predictions. This method result in two slices from each volume being becoming impossible to process and as such the \ac{FOV} would have to be increased. Alternatively multiple network architectures such as \ac{RNN} \cite{chen_combining_2016}, and \ac{LSTM} \cite{stollenga_parallel_2015} are designed to have a memory therefore enabling them to retain contextual information from slice to slice. Alternatively, if more training data were available, a 3D \ac{CNN} could be explored to ascertain if the any improved accuracy is worth the increase in hardware requirements and reduced generalisability.
 
\begin{figure}[H]
	\centering
	\includegraphics[width=0.5\textwidth]{ML/rgb_kidney.png}
	\caption{An example of using colour information to represent adjacent slices for processing by a \ac{CNN}. Vessels in the liver can clearly be seen travelling through the three slices as they change from red to green to blue.}
	\label{fig:ml_rgb_slices}	
\end{figure}

This image contrast was chosen as a result of recent publications comparing \tone- and \ttwo-weighted images for \ac{TKV} assessment reporting that \ttwo-weighted images provide better quality to enable \ac{TKV} measurements, leading to improved reproducibility with lower intra- and inter-reader variability \cite{van_gastel_t1_2018}. Other contrasts e.g. a \tone-weighted scan, could also be collected, registered to the \ttwo-weighted images and used as another channel to inform segmentation.

This network was validated on healthy subjects and \ac{CKD} patients, but has not been trained and validated on subjects with \ac{ADPKD}. These subjects have many more cysts in their kidneys, while the \ac{CNN} was able to segment cysts encountered in the \ac{CKD} cohort, it would be beneficial for future work on \ac{ADPKD} to retrain the network with \ac{HC}, \ac{CKD} and \ac{ADPKD} data, where \ac{TKV} is a recognised biomarker of disease progression.

Another common segmentation task in renal imaging is generating an \ac{ROI} for the renal cortex and medulla. There are some automated methods of achieving this once a total kidney mask has been produced \cite{cox_multiparametric_2017, morris_segmentation_2019}, however there has been no work on the application of deep learning to this task. In addition to the acquisition of the \ttwo-weighted dataset used here, a \tone-weighted dataset designed to optimise the contrast between cortex and medulla was also collected on each subject \cite{will_automated_2014}, an example of which is shown in Figure \ref{fig:ml_t1}. Using this data, it may be possible to develop this method further such that an automated mask for each tissue type is produced. 

\begin{figure}[H]
	\centering
	\includegraphics[width=.5\textwidth]{ML/T1/T1W.png}
	\caption{An example of the data collected to enable segmentation of the renal cortex and medulla.}
	\label{fig:ml_t1}
\end{figure}

\section{Conclusions}

A \ac{CNN} has been shown to successfully segment the kidneys of both \ac{HC} and \ac{CKD} subjects from \ttwo-weighted data delivering a mean Dice score of 0.93 $\pm$ 0.01 leading to a mean volume of 1.2 $\pm$ 16.2 m$\ell$ less than the manually segmented \ac{TKV} and mean surface distance of 0.65 $\pm$ 0.21 mm. The \ac{CNN} produces higher than human precision, with a \ac{CoV} and \ac{ICC} of 1.5 $\pm$ 0.5 \% and 0.993 respectively. The accuracy of the network could be further increased via the acquisition of more training data, something the renal group at \ac{SPMIC} are actively pursuing. 

The methods developed here can easily be deployed via the self contained, easy to use \ac{GUI}, thus moving renal segmentation from a 15 to 30 minute skilled task, to a 10 second task for anyone, on any hardware. Additional, this executable can be called form a terminal, making it suitable for use in scripting applications and pipelines.

Future development will focus around exploring different network architectures, use of additional contrasts and expanding the tool to produce masks of both the cortex and medulla.

\section{Acknowledgements}

We gratefully acknowledge the support of NVIDIA Corporation with the donation of the Titan Xp GPU used for this research.

\newpage
\section{References}
\defbibheading{bibliography}[\refname]{}
\printbibliography