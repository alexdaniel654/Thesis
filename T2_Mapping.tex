\chapter{Assessment of Renal $T_2$ Mapping Methods}
\label{chap:t2_mapping}

\begin{abstract}
	This work was presented as an aural presentation at the \ac{ISMRM} 28th Annual Meeting (2020) \cite{daniel_comparison_2020}.
	
	\lipsum[1]
\end{abstract}
\newpage
\acresetall

\section{Introduction}
\label{sec:t2_intro}
Quantitative \ac{MRI} is the process of taking measurements where the voxel values have numerical significance rather than simply representing signal intensity in arbitrary units \cite{tofts_quantitative_2003}. These numerically significant values can take the form of macroscale properties such as rate of oxygen consumption and blood vessel flow rates or microscale properties such as tissue \tone and susceptibility. When interpreted these values can be used improve diagnostic and treatment of patients.

The kidneys are structurally and functionally complex organs and as such lend themselves to the wide variety of MRI protocols designed to probe different aspects of the tissue and processes carried out within. While high resolution images of the kidneys morphology and basic measures such as \ac{TKV} can be very useful in diagnosing and monitoring disease progression \cite{buchanan_quantitative_2019, chapman_kidney_2012, gong_relationship_2012}, these do not fully leverage the quantitative nature of \ac{MRI}. Measurements of \tone have been shown to correlate well with fibrosis in the myocardium \cite{bull_human_2013, ferreira_t1_2013}, liver \cite{hoad_study_2015, luetkens_quantification_2018} and kidneys \cite{friedli_new_2016} and more generally an increase in \tone is associated with \ac{CKD} \cite{gillis_non-contrast_2016, cox_multiparametric_2017, buchanan_quantitative_2019}. \ac{ASL} techniques can be used to quantify renal perfusion in physiological units (mL/100g/min) and has been shown to be correlated with allograft function post renal transplant in addition to cold ischemia time and the recipients \ac{eGFR} \cite{hueper_functional_2015, artz_arterial_2011, ren_evaluation_2016, niles_longitudinal_2016}. Additionally \ac{ASL} has been used to measure a decrease in perfusion in \ac{CKD} subjects \cite{gillis_non-contrast_2016, rossi_histogram_2012, tan_renal_2014}. These techniques have proved useful when used individually however they can be combined and used in the same scanning session to greater effect as a multiparametric protocol \cite{buchanan_quantitative_2019, cox_multiparametric_2017, eckerbom_multiparametric_2019, schley_multiparametric_2018, hueper_kidney_2016}.

\ttwo mapping has found wide use in cardiac \ac{MRI} for assessment of myocardial edema \cite{gouya_rapidly_2008, giri_t2_2009, nasenstein_cardiac_2014} and iron overload \cite{guo_myocardial_2009, krittayaphong_detection_2017}. It has also effectively been used in the brain to study multiple sclerosis \cite{neema_t1-_2007}, epilepsy \cite{rugg-gunn_whole-brain_2005}, dementia \cite{knight_quantitative_2016} and Parkinson’s disease \cite{vymazal_t1_1999}. Despite these developments elsewhere in the body, \ttwo mapping has had limited uptake in the renal community.

Renal \ttwo mapping has seen most research focusing on repeatability measures \cite{de_bazelaire_mr_2004, zhang_reproducibility_2011, li_measuring_2015, de_boer_multiparametric_2020} with clinical uses in the field of assessment of allograft function in mice \cite{hueper_kidney_2016} and humans \cite{mathys_t2_2011, adams_multiparametric_2020}, and has shown potential for early diagnosis of \ac{ADPKD} \cite{franke_magnetic_2017} and assessment of clear cell renal cell carcinoma \cite{adams_use_2019}.

In the existing literature, there is a substantial variation in quoted \ttwo values for the kidneys of healthy volunteers, this is thought to be, in part, due to the differences in \ttwo mapping methodologies. There are currently four main methods, a basic spin echo method, a multi echo-spin echo method, a gradient spin echo method and a \ac{CPMG} \ttwo prep method. Here we aim to compare each of these methods in the context of renal \ttwo mapping to ascertain which is most suitable. This involves evaluating each methods quantitative accuracy, image quality, susceptibility to flow and suitability for use in-vivo.

\section{Methods}
\label{sec:t2_methods}

\subsection{Data Acquisition}
\label{subsec:t2_acq_schemes}

All data was acquired on a 3T Philips Ingenia system (Philips Medical Systems, Best, The Netherlands). The 14 element \ttwo array of a QalibreMD System Standard Model 130 containing spheres doped with varying concentrations of \ce{MnCl_2} to modulate \ttwo between 5 ms and 650 ms was used to compare the accuracy of \ttwo measurements to a known ground truth, Figure \ref{fig:t2_phantom_schematic}. Additionally, a square grid etched into plate three of the phantom was used to assess the degree of image blurring. 

\begin{figure}[H]
	\centering
	\includegraphics[width=0.5\textwidth]{T2_Mapping/Phantom_Example/T2_Spheres.eps}
	\caption{A schematic of the $T_2$ spheres in the QalibreMD phantom.}
	\label{fig:t2_phantom_schematic}	
\end{figure}

To investigate the effects of flow upon \ttwo measurements, a Gold Standard Phantoms \ac{QASPER} phantom was used. This allows the \ttwo of the perfusate to be measured at rest and whilst being pumped through the phantom at a range of flow rates, Figure \ref{fig:t2_flow_phantom_schematic}. 

\begin{figure}[H]
	\centering
%	\includegraphics[width=0.5\textwidth]{T2_Mapping/Phantom_Example/T2_Spheres.eps}
	\missingfigure{Flow phantom schematic}
	\caption{A schematic of \ac{QASPER} phantom used to quantify the effects of flow upon the \ttwo measurements.}
	\label{fig:t2_flow_phantom_schematic}	
\end{figure}

Both phantoms were scanned using a 32-channel head coil. All data acquired on human subjects was done with approval of the local ethics committee and the study was conducted in accordance with the Helsinki Declaration. The subjects gave written, informed consent. Humans were scanned using a 16-channel anterior coil array and 16-channel posterior coil array. The study cohort consisted of 5 healthy participants (2 female, 3 male, mean age 31 $\pm$ 8).

The protocol consisted of a survey, localisers, $B_0$ and $B_1$ mapping, then each of the optimised \ttwo mapping sequences. In-vivo subjects also had \ttwo-weighted and \tone-weighted structural scans to enable segmentation of the whole kidneys, and cortex/medulla respectively \cite{petzold_building_2014, will_automated_2014}, these \ac{ROI} are then used to calculate the mean \ttwo of each tissue type. A summary of the parameters of each \ttwo mapping sequence is shown in Table \ref{tab:t2_sequence_overview}. Each protocol was designed to be approximately two minutes (before respiratory triggering) and keep key parameters such as voxel size and \ac{FOV} constant.

\begin{table}[H]
	\centering
	\begin{adjustbox}{width=1.2\textwidth, center}
	\begin{tabularx}{1.25\textwidth}{X|X|X|X|X}
		                                                     & Spin Echo - Echo Planar Imaging & Multi-Echo Turbo Spin Echo & Gradient Spin Echo & CPMG \ttwo Prep EPI \\ \hline
		Abbreviation                                         & SE-EPI                          & ME-TSE                     & GraSE              & CPMG \ttwo Prep     \\ \hline
		TE (min:step:max) (ms)                               & 20:10:70                        & 13:13:130                  & 11.2:5.6:173.3     & 0:20:160            \\ \hline
		Number of   echoes                                   & 6                               & 10                         & 30                 & 9                   \\ \hline
		TR (ms)                                              & 5000                            & 3000                       & 3000               & 3000                \\ \hline
		Voxel Size   (mm$^3$)                                & 3 $\times$ 3 $\times$ 5         & 3 $\times$ 3 $\times$ 5    &3 $\times$ 3 $\times$ 5 &3 $\times$ 5.65 $\times$ 5\\ \hline
		FoV (mm$^3$)                                         & 288 $\times$ 288 $\times$ 25    & 288 $\times$ 288 $\times$ 25   &288 $\times$ 288 $\times$ 25&288 $\times$ 288 $\times$ 25\\ \hline
		Signal   Averages                                    & 2                               & 1                          & 1                  & 1                   \\ \hline
		Acquisition Mode                                     & Multi Slice                     & Multi Slice                & Multi Slice        & Multiple 2D         \\ \hline
		Fast   Imaging Mode                                  & EPI                             & TSE                        & GraSE              & TFEPI               \\ \hline
		Flip Angle                                           & 90$\degree$                     & 90$\degree$                & 90$\degree$        & 90$\degree$         \\ \hline
		Bandwidth   (Hz)                                     & 40                              & 180                        & 405                & 113                 \\ \hline
		SENSE                                                & 2.55                            & 2.55                       & 2.55               & 3                   \\ \hline
		Halfscan                                             & 0.838                           & No                         & No                 & 0.706               \\ \hline
		TSE Factor                                           & N/A                             & 10                         & 30                 & N/A                 \\ \hline
		EPI Factor                                           & 37                              & N/A                        & 3                  & 17                  \\ \hline
		Respiratory Compensation                             & Triggered                       & Triggered                  & Triggered          & Triggered           \\ \hline
		Acquisition Time \tiny{(before respiratory compensation)} & 3 min 0 sec                & 1 min 57 sec               & 2 min 6 sec        & 2 min 23 sec  
	\end{tabularx}
	\end{adjustbox}
	\caption{A summary of the acquisition parameters of each of the \ttwo mapping methods compared.}
	\label{tab:t2_sequence_overview}
\end{table}

\subsubsection{Spin Echo-Echo Planar Imaging}
This technique is the simplest of the four consisting of a 90$\degree$ excitation pulse, followed by a 180$\degree$ \ac{RF} pulse \ac{TE}/2 ms later, leading to an echo at \ac{TE}. This 180$\degree$ pulse corrects for components of the signal lost due to static field inhomogeneities however does not correct for \ttwo effects therefore by repeating the sequence multiple times with different \ac{TE}, the \ttwo decay can be sampled. An \ac{EPI} readout is used to sample the signal during the echo. This sequence suffers from a relatively low \ac{SNR}, hence the two signal averages; this means that only six different echo times are recorded. An overview of the sequence is shown in Figure \ref{fig:t2_se-epi_seq}.

\begin{figure}[H]
	\centering
	\includegraphics[width=0.7\textwidth]{T2_Mapping/Pulse_Diagrams/T2_SE_EPI.eps}
	\caption{A pulse sequence diagram of the \ac{SE}-\ac{EPI} scheme.}
	\label{fig:t2_se-epi_seq}	
\end{figure}

\subsubsection{Multi-Echo Turbo Spin Echo}
This sequence also uses a multi-slice spin echo; however the \ac{EPI} readout is replaced by a \ac{TSE} readout. This replaces the singe 180$\degree$ pulse with a train of pulses with an echo forming between each allowing the whole \ttwo decay to be sampled in a single echo train. Different echo times are sampled by varying the number and spacing of the 180$\degree$ pulses. The decrease in acquisition time per echo compared to the \ac{SE}-\ac{EPI} sequence meant that ten echoes were collected per \ttwo map at the minimum possible echo spacing, 13 ms. A schematic of the \ac{PSD} can be seen in Figure \ref{fig:t2_me-tse_seq}.

\begin{figure}[H]
	\centering
	\includegraphics[width=0.7\textwidth]{T2_Mapping/Pulse_Diagrams/T2_ME_TSE.eps}
	\caption{A pulse sequence diagram of the \ac{ME-TSE} scheme.}
	\label{fig:t2_me-tse_seq}	
\end{figure}

\subsubsection{Gradient Spin Echo}

To achieve further acceleration over the \ac{ME-TSE} sequence, a \ac{GraSE} sequence can be used. Here two gradient echoes are collected for every spin echo with the spin echo and gradient echoes being used for the acquisition of the centre and periphery of k-space respectively. The multiple $k$-space profiles collected per spin echo enables a decrease in the echo spacing of the sequence compared to the \ac{ME-TSE}, decreasing to 5.6 ms and thus thirty echoes are collected per \ttwo map. The \ac{PSD} for the \ac{GraSE} acquisition is shown in Figure \ref{fig:t2_grase_seq}.

\begin{figure}[H]
	\centering
	\includegraphics[width=0.7\textwidth]{T2_Mapping/Pulse_Diagrams/T2_GraSE.eps}
	\caption{A pulse sequence diagram of the \ac{GraSE} scheme.}
	\label{fig:t2_grase_seq}	
\end{figure}

\subsubsection{CPMG $T_2$ Preparation}

The \ac{CPMG} \ttwo preparation sequence is considered the gold standard in terms of accuracy. This sequence consists of the 90$\degree$ excitation pulse to transfer the magnetisation into the transverse plane as with the other sequences. A series of spatially non-selective 180$\degree$ pulses with alternating phases are then applied; by varying the number and temporal spacing, $\tau_{\textup{CPMG}}$, of these pulses the degree of \ttwo weighting can me modulated to achieve different \ac{eTE}. An \ac{EPI} readout scheme is then used to sample the signal. An overview of this sequence is shown in Figure \ref{fig:t2_cpmg_t2prep_seq}.

\begin{figure}[H]
	\centering
	\includegraphics[width=0.7\textwidth]{T2_Mapping/Pulse_Diagrams/T2_T2_Prep.eps}
	\caption{A pulse sequence diagram of the \ac{CPMG} $T_2$ preperation scheme.}
	\label{fig:t2_cpmg_t2prep_seq}	
\end{figure}


\subsection{Post Processing}

All post processing was performed using Python 3.7 making use of the \ac{UKAT} toolbox \cite{nery_ukrin_2020}. All curve fitting uses a least squares trust region reflective method to estimate the variables in the desired function.

\subsubsection{\ttwo Fitting Methods}

The data were fit using a mono-exponential model, however, there are multiple models for fitting noisy data, each of which was evaluated here and illustrated in Figure \ref{fig:t2_fitting_methods} The basic fit simply takes the signal from each voxel at each \ac{TE} and fits it to 
\begin{equation}
	S(t) = S_0 \cdot e^{-t/T_2}.
	\label{eq:t2}
\end{equation}
If no noise were present in the data, this would be the optimum method however, decreased SNR of later \ac{TE} often leads to inaccurate fits. To combat this, the equation the data is fit to can be modified to
\begin{equation}
	S(t) = S_0 \cdot e^{-t/T_2} + \epsilon
	\label{eq:t2_noise}
\end{equation}
where $\epsilon$ represents thermal noise and a baseline in the signal due to long \ttwo compartments, this fitting method is referred to as ``noise fit''. Another common method of negating the effects of the low \ac{SNR} later \ac{TE} is to discard data below a threshold, illustrated here at 0.2 AU and referred to as ``discard fit''. Finally, a combination of both noise estimation and discarding was performed (discard and noise fit).

\begin{figure}[H]
	\centering
	\includegraphics[width=0.7\textwidth]{T2_Mapping/fitting_methods.pdf}
	\caption{Each of the four fitting methods being used to estimate \ttwo of the same simulated data.}
	\label{fig:t2_fitting_methods}	
\end{figure}

These fitting methods can either be applied on a voxel-by-voxel basis to generate spatial maps or the signal from all voxels in an \ac{ROI} with a single \ttwo can be averaged at each \ac{TE} with parameters being fit to the subsequent mean signal. 

\subsection{Assessment of Data}

\subsubsection{Quantifying Accuracy}
Using the QalibreMD phantom, the quantitative accuracy of each of the sequences was assessed. A \ac{ROI} was defined for each of the spheres in the \ttwo array and the mean of all voxels within the sphere at each \ac{TE} calculated. The estimated values of \ttwo were compared to the ground truth literature values and their discrepancy assessed over both the full range of \ttwo in the array (5 ms – 650 ms) and the range of \ttwo reported in the kidneys (40 ms – 200 ms) \cite{wolf_magnetic_2018}. Accuracy was summarised over these ranges using \ac{MRE} defined as 
\begin{equation}
	\textup{MRE} = \frac{1}{N}\sum_{n}^{i=1}\left|  \frac{t_{2\; i}^{\textup{ground truth}} - t_{2\; i}^{\textup{estimate}}}{t_{2\; i}^{\textup{ground truth}}}\right|.
	\label{eq:t2_mre}
\end{equation}

\subsubsection{Effects of Flow}
The kidneys are highly perfused organ, as such, the effects of fluid flow through the area being imaged should be evaluated. This was achieved using a Gold Standard Phantoms \ac{QASPER} phantom. This phantom comprises of a \ac{MRI} compatible pump with adjustable continuous flow rates from 0 m$\ell$/min to 350 m$\ell$/min, the perfusate exiting this pump continues into a series of simulated arterioles before entering a porous media designed to simulate a capillary bed. The porous media is imaged using each sequence over the full range of flow rates the phantoms pump can deliver. \ttwo maps are calculated and the mean \ttwo calculated to evaluate the robustness of each sequence to variations in perfusion.

\subsubsection{Blurring}
Unfortunately \ac{MRI} doesn't produce perfect images, every signal is subject to a degree of blurring or spreading out into surrounding voxels. The amount of this blurring is different for each sequence and can dramatically effect the readability of an image and ultimately its clinical utility. In \ac{MRI} the amount and characteristics of the blur are usually spatially invariant, that is to say, if a voxel in the centre of an image is blurred over its five neighbouring voxels in the phase encode direction, a voxel at the edge of the image would have the same five voxel blur applied to it. We wish to quantify the amount of blurring produced by each of the sequences outlined in \ref{subsec:t2_acq_schemes}.

The observed image, $h$, can be modelled as a the ideal, unblurred signal, $f$ distorted by a filter, $g$, Figure \ref{fig:t2_1d_blur}. This distorting filter is known as the \ac{PSF} and is the theoretical signal produced by an infinitely small point source object or, in practice, the blurring observed in the imaging system produced when an object much smaller than the systems resolving power is imaged. In a spatially invariant system such as \ac{MRI} the recorded signal is simply a convolution of the true signal and the \ac{PSF} i.e. $f \ast g = h$. By fitting a Gaussian to the \ac{PSF} we can quantify the degree of blurring in the image \cite{chaimow_more_2017, chaimow_more_2017-1}. 
\begin{figure}[H]
	\centering
	\includegraphics[width=0.8\textwidth]{T2_Mapping/1D_Blur/1d_conv.eps}
	\caption{The convolution of the ideal signal, $f$, and the \ac{PSF}, $g$, produces the measured signal, $h$.}
	\label{fig:t2_1d_blur}	
\end{figure}

A grid etched into one of the plastic plates of the QalibreMD phantom was imaged using each of the \ttwo mapping methods with an echo time of 20 ms and a 0.5 mm$^3$ isotropic structural scan collected. The resolution of the structural scan is much greater than the resolution of the \ttwo mapping scans, therefore it can be seen as an approximation of the ideal image produced by each \ttwo mapping method. This allows a deconvolution of the \ac{PSF} from the \ttwo mapping scans and, by fitting a Gaussian to the line profiles of the \ac{PSF} in each direction, estimate the \ac{FWHM} of the \ac{PSF} as shown in Figure \ref{fig:t2_2d_blur}. Quoted values are the maximum \ac{PSF} of each of the two directions as this is the limiting factor in an images readability, for the example in Figure \ref{fig:t2_2d_blur}, the quoted \ac{PSF} would be 9.49 $\pm$ 0.23 mm.

\begin{figure}[H]
	\centering
	\includegraphics[width=0.9\textwidth]{T2_Mapping/2D_Blur/Example/2D_Blur.eps}
	\caption{An overview of the estimation of the \ac{PSF}. The \ttwo weighted data has had additional blur added to make the effect of each processing step clearer.}
	\label{fig:t2_2d_blur}	
\end{figure}

\subsubsection{In-Vivo}

Using the \tone-weighted structural scans \ac{ROI} were defined for both the renal cortex and medulla of each subject. These \ac{ROI} were the applied to \ttwo maps generated using each method and a mean and standard deviation of \ttwo for each tissue type calculated. \ttwo maps were also qualitatively assessed.

\section{Results}

\subsection{Fitting Methods}

Each fitting method was tested on data acquired on the QalibreMD phantom and in-vivo. A full breakdown of the accuracy of each fitting method, applied to each sequence, over different ranges of \ttwo can be seen in Table \ref{tab:t2_phantom_fitting_methods} and Figure \ref{fig:t2_in_vivo_fitting_methods}. The designers of the phantom calculate the mean signal from each sphere then fit to a \ttwo decay, resulting in a quicker and more accurate measurement of the homogenous \ttwo of each sphere \cite{mristandards_mristandardsphantomviewer_2020}. Due to the heterogeneity within the kidneys, this method could not be applied in-vivo and as such, the accuracy of the sequences and fitting methods was evaluated by performing a voxel-by-voxel fit, then calculating the mean \ttwo from an \ac{ROI} of each sphere in the resulting map.

\begin{table}[H]
	\centering
	\begin{adjustbox}{width=1.2\textwidth, center}
	\begin{tabularx}{1.3\textwidth}{X|X|X|X|X|X|X|X|X|X}
		\ttwo   Range            & \multicolumn{4}{c|}{MRE   (5 ms – 650 ms) (\%)}                         & \multicolumn{5}{c}{MRE   (40 ms – 200 ms) (\%)}              \\ \hline
		Fitting   Method      & Basic     & Noise     & Discard   & Discard   and Noise & \multicolumn{2}{l|}{Basic}     & Noise      & Discard   & Discard   and Noise \\ \hline
		SE-EPI                & 36 $\pm$  34   & 202 $\pm$  437 & 33 $\pm$  34   & 238 $\pm$  463           & \multicolumn{2}{l|}{8 $\pm$  5}     & 32 $\pm$  42    & 22 $\pm$  8    & 31 $\pm$  42             \\ \hline
		ME-TSE                & 38   $\pm$  31 & 13   $\pm$  16 & 41   $\pm$  35 & 35   $\pm$  36           & \multicolumn{2}{l|}{23   $\pm$  13} & 14   $\pm$  3   & 15   $\pm$  13 & 13   $\pm$  4            \\ \hline
		GraSE                 & 32 $\pm$  29   & 23 $\pm$  27   & 26 $\pm$  28   & 33 $\pm$  37             & \multicolumn{2}{l|}{15 $\pm$  4}    & 11 $\pm$  5     & 13 $\pm$  4    & 9 $\pm$  7               \\ \hline
		CPMG   T2 Prep        & 18   $\pm$  15 & 30   $\pm$  53 & 20   $\pm$  18 & 28   $\pm$  28           & \multicolumn{2}{l|}{11   $\pm$  1}  & 8   $\pm$  5    & 11   $\pm$  1  & 6   $\pm$  5             \\ \hline
		Mean   over sequences & 31   $\pm$  8  & 67   $\pm$  90 & 30   $\pm$  9  & 83   $\pm$  103          & \multicolumn{2}{l|}{14   $\pm$  7}  & 16   $\pm$  11  & 14   $\pm$  36 & 15   $\pm$  11          
	\end{tabularx}
	\end{adjustbox}
	\caption{\ac{MRE} when measuring \ttwo of the QalibreMD phantom over different ranges using each sequence and fitting method. 5 ms – 650 ms is the full range of \ttwo available in the phantom and 40 ms – 200 ms is the range of \ttwo reported in the kidneys.}
	\label{tab:t2_phantom_fitting_methods}
\end{table}

\begin{figure}[H]
	\centering
	\begin{subfigure}[c]{0.47\textwidth}
		\centering
		\includegraphics[width=1\textwidth]{T2_Mapping/Fitting_methods_invivo/basic.eps}
		\caption{Basic fit}
		\label{fig:t2_in_vivo_fitting_methods_basic}
	\end{subfigure}
	\hfill
	\begin{subfigure}[c]{0.47\textwidth}
		\centering
		\includegraphics[width=1\textwidth]{T2_Mapping/Fitting_methods_invivo/noise.eps}
		\caption{Noise fit}
		\label{fig:t2_in_vivo_fitting_methods_noise}
	\end{subfigure}

	\vskip\baselineskip
	\begin{subfigure}[c]{0.47\textwidth}
		\centering
		\includegraphics[width=1\textwidth]{T2_Mapping/Fitting_methods_invivo/discard.eps}
		\caption{Discard fit}
		\label{fig:t2_in_vivo_fitting_method_discards}
	\end{subfigure}
	\hfill
	\begin{subfigure}[c]{0.47\textwidth}
		\centering
		\includegraphics[width=1\textwidth]{T2_Mapping/Fitting_methods_invivo/noise_discard.eps}
		\caption{Discard and noise fit}
		\label{fig:t2_in_vivo_fitting_methods_noise_discard}
	\end{subfigure}
	\caption{\ttwo maps generated using each of the fitting methods using the \ac{GraSE} sequence. The reduction in longer \ttwo values when fit with a noise term observed in the phantom can be seen here in the decreased \ttwo of the kidneys, while the spleen remains relatively similar. Discarding has relatively little effect on the kidneys but does increase the variance in \ttwo within the cortex and medulla compared to the basic fit.}
	\label{fig:t2_in_vivo_fitting_methods}
\end{figure}

The noise fit results in an increase accuracy compared to the basic fit when measuring the \ttwo of the shortest \ttwo spheres, especially for sequences with a short echo spacing where the majority of echoes in the signal are after the sphere has fully relaxed back to its baseline noise level. This increase in accuracy for short \ttwo is at the expense of the accuracy of the long \ttwo spheres. The combination of decreased dynamic range and an extra parameter to optimise resulted in an inaccurate characterisation of these spheres, especially of the \ac{SE}-\ac{EPI} sequence due to the short final echo time and thus lower dynamic range. In the range of \ttwo reported in the kidney, the noise fit increased the average error over the four sequences. 

The discard fit requires an empirical threshold to be chosen. If chosen correctly, this resulted in slightly improved accuracy however if the threshold was not correctly chosen the accuracy of \ttwo was compromised. While optimising the threshold is trivial when the known reference values are available, this is more difficult in-vivo and therefore, given the increase in accuracy was marginal and only effected spheres of short \ttwo, consistent results were deemed preferable.

The combination of discard and noise fit resulted in a decreased accuracy from the basic fit. Spheres that benefited from the additional noise term relax to their baseline noise quickly, however by discarding these echoes, the estimate of noise becomes inaccurate and thus so does the estimate of \ttwo.

It was therefore concluded that a basic fit should be used for all subsequent renal data.

\subsection{Phantom Verification}

\subsubsection{Accuracy}

Each sequence was used to image the T2 array in the QalibreMD phantom, Figure \ref{fig:t2_phantom_loc}. Figure \ref{fig:t2_phantom_cor} shows the measured \ttwo plot against the reference \ttwo for each method. All methods struggle to accurately measure the very short \ttwo spheres in the array with the \ac{CPMG} \ttwo prep method faring best. The \ac{SE}-\ac{EPI} method overestimates short \ttwo spheres due to the first \ac{TE} being sampled at 20 ms but is also underestimating long \ttwo because of its small range in \ac{TE}. When considering only the spheres with physiologically similar \ttwo to the kidneys, the \ac{SE}-\ac{EPI} method is the most accurate with the \ac{GraSE} and \ac{CPMG} \ttwo prep delivering similar results. This is mirrored by the \ac{MRE} shown in Table \ref{tab:t2_phantom_acquisition_methods}. Example \ttwo maps of the phantoms \ttwo array imaged with each sequence are shown in Figure \ref{fig:t2_phantom_maps}.

\begin{figure}[H]
	\centering
	\includegraphics[width=0.6\textwidth]{T2_Mapping/Phantom_Example/Localiser_multi.eps}
	\caption{The $T_2$ spheres inside the QalibreMD phantom.}
	\label{fig:t2_phantom_loc}	
\end{figure}

\begin{figure}[H]
	\centering
	\begin{subfigure}[c]{0.47\textwidth}
		\centering
		\includegraphics[width=1\textwidth]{T2_Mapping/Phantom_cor/basic_fit_full.pdf}
		\caption{}
		\label{fig:t2_phantom_cor_full}
	\end{subfigure}
	\hfill
	\begin{subfigure}[c]{0.47\textwidth}
		\centering
		\includegraphics[width=1\textwidth]{T2_Mapping/Phantom_cor/basic_fit_kidney.pdf}
		\caption{}
		\label{fig:t2_phantom_cor_kidney}
	\end{subfigure}
	\caption{\ttwo measured using each method compared to the reference \ttwo from literature. (\subref{fig:t2_phantom_cor_full}) The full range of \ttwo spheres is shown on logarithmic axis with the range of \ttwo reported in the kidneys shaded in red (\subref{fig:t2_phantom_cor_kidney}) The spheres with \ttwo in the range of the kidneys are shown on linear axis.}
	\label{fig:t2_phantom_cor}
\end{figure}

\begin{table}[H]
	\centering
	\begin{tabular}{l|l|l}
		Acquisition   Method & MRE   (5 ms – 650 ms) (\%) & MRE   (40 ms – 200 ms) (\%) \\ \hline
		SE-EPI               & 36 $\pm$ 34                    & 8   $\pm$ 5                     \\ \hline
		ME-TSE               & 38   $\pm$ 31                  & 23 $\pm$ 13                     \\ \hline
		GraSE                & 32 $\pm$ 29                    & 15   $\pm$ 4                    \\ \hline
		CPMG   \ttwo Prep       & 18   $\pm$ 15                  & 11   $\pm$ 1                   
	\end{tabular}
	\caption{\ac{MRE} when measuring \ttwo of the QalibreMD phantom over different ranges using each sequence. 5 ms – 650 ms is the full range of \ttwo available in the phantom and 40 ms – 200 ms is the range of \ttwo expected in the kidneys.}
	\label{tab:t2_phantom_acquisition_methods}
\end{table}

\begin{figure}[H]
	\centering
	\begin{subfigure}[c]{0.47\textwidth}
		\centering
		\includegraphics[width=1\textwidth]{T2_Mapping/Phantom_maps/SE.pdf}
		\caption{\ac{SE}-\ac{EPI}}
		\label{fig:t2_phantom_map_se}
	\end{subfigure}
	\hfill
	\begin{subfigure}[c]{0.47\textwidth}
		\centering
		\includegraphics[width=1\textwidth]{T2_Mapping/Phantom_maps/ME.pdf}
		\caption{\ac{ME-TSE}}
		\label{fig:t2_phantom_map_me_tse}
	\end{subfigure}
	
	\vskip\baselineskip
	\begin{subfigure}[c]{0.47\textwidth}
		\centering
		\includegraphics[width=1\textwidth]{T2_Mapping/Phantom_maps/GraSE.pdf}
		\caption{\ac{GraSE}}
		\label{fig:t2_phantom_map_grase}
	\end{subfigure}
	\hfill
	\begin{subfigure}[c]{0.47\textwidth}
		\centering
		\includegraphics[width=1\textwidth]{T2_Mapping/Phantom_maps/CPMG.pdf}
		\caption{\ac{CPMG} \ttwo Prep}
		\label{fig:t2_phantom_map_cpmg_t2_prep}
	\end{subfigure}
	\caption{\ttwo maps of the QaliberMD system phantom \ttwo array generated using each sequence.}
	\label{fig:t2_phantom_maps}
\end{figure}

\subsubsection{Sensitivity to Flow}

\subsubsection{Image Quality}

\subsection{In-Vivo}

$T_2$ maps using all four methods were collected on the same subject in the same scanning session to allow for a direct comparison of the in-vivo data. 

\begin{figure}[H]
	\centering
	\begin{subfigure}[c]{0.9\textwidth}
		\centering
		\begin{subfigure}[c]{0.47\textwidth}
			\centering
			\includegraphics[width=1\textwidth]{T2_Mapping/SE/SE_Raw_Echoes.eps}
			\caption{}
			\label{fig:t2_t2_se_raw}
		\end{subfigure}
		\hfill
		\begin{subfigure}[c]{0.47\textwidth}
			\centering
			\includegraphics[width=1\textwidth]{T2_Mapping/SE/SE_Map.eps}
			\caption{}
			\label{fig:t2_t2_se_map}
		\end{subfigure}
	\end{subfigure}
	\vskip\baselineskip
	\begin{subfigure}[c]{0.9\textwidth}
		\centering
		\includegraphics[width=1\textwidth]{T2_Mapping/SE/SE_Decay.eps}
		\caption{}
		\label{fig:t2_t2_se_decay}			
	\end{subfigure}
	\caption{(\subref{fig:t2_t2_se_raw}) The raw data used to generate the \ac{SE}-\ac{EPI} $T_2$ map.  (\subref{fig:t2_t2_se_map}) An example slice from the \ac{SE}-\ac{EPI} $T_2$ map. (\subref{fig:t2_t2_se_decay}) The signal decay for the renal cortex and medulla.} 
	\label{fig:t2_t2_se}
\end{figure}

The \ac{SE}-\ac{EPI} method (Figure \ref{fig:t2_t2_se}) generated maps with little blurring however there is also a lack of differentiation in $T_2$ between the renal cortex and medulla. The data collected at \ac{TE} of 20 ms appears to be artificially high and leads to a reduction in fit $T_2$. This sequence is the most susceptible of the methods to patient motion due to the acquisition method of a series per \ac{TE}, this increase in motion is clear when scrolling through \ac{TE}. 

\begin{figure}[H]
	\centering
	\begin{subfigure}[c]{0.9\textwidth}
		\centering
		\begin{subfigure}[c]{0.47\textwidth}
			\centering
			\includegraphics[width=1\textwidth]{T2_Mapping/ME/ME_Raw_Echoes.eps}
			\caption{}
			\label{fig:t2_t2_me_raw}
		\end{subfigure}
		\hfill
		\begin{subfigure}[c]{0.47\textwidth}
			\centering
			\includegraphics[width=1\textwidth]{T2_Mapping/ME/ME_Map.eps}
			\caption{}
			\label{fig:t2_t2_me_map}
		\end{subfigure}
	\end{subfigure}
	\vskip\baselineskip
	\begin{subfigure}[c]{0.9\textwidth}
		\centering
		\includegraphics[width=1\textwidth]{T2_Mapping/ME/ME_Decay.eps}
		\caption{}
		\label{fig:t2_t2_me_decay}			
	\end{subfigure}
	\caption{(\subref{fig:t2_t2_me_raw}) The raw data used to generate the \ac{ME-TSE} $T_2$ map.  (\subref{fig:t2_t2_me_map}) An example slice from the \ac{ME-TSE} $T_2$ map. (\subref{fig:t2_t2_me_decay}) The signal decay for the renal cortex and medulla.} 
	\label{fig:t2_t2_me}
\end{figure}

The map generated by the \ac{ME-TSE} method (Figure \ref{fig:t2_t2_me}) suffers from a large amount of blurring due to the relatively long echo train length. The number of echoes acquired is limited to the \ac{TSE} factor therefore to acquire ten echoes, a \ac{TSE} factor of ten needs to be used. This blurring leads to structures being obscured in the map and only a very small differentiation between cortex and medulla.

\begin{figure}[H]
	\centering
	\begin{subfigure}[c]{0.9\textwidth}
		\centering
		\begin{subfigure}[c]{0.47\textwidth}
			\centering
			\includegraphics[width=1\textwidth]{T2_Mapping/GraSE/GraSE_Raw_Echoes.eps}
			\caption{}
			\label{fig:t2_t2_grase_raw}
		\end{subfigure}
		\hfill
		\begin{subfigure}[c]{0.47\textwidth}
			\centering
			\includegraphics[width=1\textwidth]{T2_Mapping/GraSE/GraSE_Map.eps}
			\caption{}
			\label{fig:t2_t2_grase_map}
		\end{subfigure}
	\end{subfigure}
	\vskip\baselineskip
	\begin{subfigure}[c]{0.9\textwidth}
		\centering
		\includegraphics[width=1\textwidth]{T2_Mapping/GraSE/GraSE_Decay.eps}
		\caption{}
		\label{fig:t2_t2_grase_decay}			
	\end{subfigure}
	\caption{(\subref{fig:t2_t2_grase_raw}) The raw data used to generate the \ac{GraSE} $T_2$ map.  (\subref{fig:t2_t2_grase_map}) An example slice from the \ac{GraSE} $T_2$ map. (\subref{fig:t2_t2_grase_decay}) The signal decay for the renal cortex and medulla.}
	\label{fig:t2_t2_grase}
\end{figure}

Using the \ac{GraSE} method the data in Figure \ref{fig:t2_t2_grase} was collected. There is a clear difference between cortical and medullary $T_2$ and the data fits well to a $T_2$ decay (Figure \ref{fig:t2_t2_grase_decay}). The signal from the first echo in Figure \ref{fig:t2_t2_grase_decay} is too intense, this effect was even more pronounced when no startup echoes were used. For tissues with a longer $T_2$ using two startup echoes would be preferable however this makes measurements of tissues with a short $T_2$ more inaccurate, as such a compromise of a single startup echo was used. The short echo-spacing made possible by \ac{GraSE} means more \ac{TE} can be sampled and therefore leads to a more accurate fit.

\begin{figure}[H]
	\centering
	\begin{subfigure}[c]{0.9\textwidth}
		\centering
		\begin{subfigure}[c]{0.47\textwidth}
			\centering
			\includegraphics[width=1\textwidth]{T2_Mapping/T2prep/T2prep_Raw_Echoes.eps}
			\caption{}
			\label{fig:t2_t2_t2prep_raw}
		\end{subfigure}
		\hfill
		\begin{subfigure}[c]{0.47\textwidth}
			\centering
			\includegraphics[width=1\textwidth]{T2_Mapping/T2prep/T2prep_Map.eps}
			\caption{}
			\label{fig:t2_t2_t2prep_map}
		\end{subfigure}
	\end{subfigure}
	\vskip\baselineskip
	\begin{subfigure}[c]{0.9\textwidth}
		\centering
		\includegraphics[width=1\textwidth]{T2_Mapping/T2prep/T2prep_Decay.eps}
		\caption{}
		\label{fig:t2_t2_t2prep_decay}			
	\end{subfigure}
	\caption{(\subref{fig:t2_t2_t2prep_raw}) The mean data at each \ac{eTE} used to generate the $T_2$ preparation $T_2$ map.  (\subref{fig:t2_t2_t2prep_map}) An example slice from the $T_2$ preparation $T_2$ map. (\subref{fig:t2_t2_t2prep_decay}) The signal decay for the renal cortex and medulla.} 
	\label{fig:t2_t2_t2prep}
\end{figure}

The map made using the $T_2$ preparation method (Figure \ref{fig:t2_t2_t2prep}) suffers from noise in the raw data, this is despite there being three acquisitions at each \ac{eTE}. When comparing Figure \ref{fig:t2_t2_t2prep_map} to Figure \ref{fig:t2_t2_grase_map} it's possible to see that some of the areas of greater $T_2$ do match with the medulla, however the degree of noise in \ref{fig:t2_t2_t2prep_map} means it is un-usable on its own. The small number of \ac{eTE} collected means the uncertainty in the fit $T_2$ is higher for this method.

The two methods that have delivered the highest image quality, \ac{SE}-\ac{EPI} and \ac{GraSE}, produce substantially different values of $T_2$ in-vivo. Even when the data from the 20 ms volume is omitted from the \ac{SE}-\ac{EPI} fit, the $T_2$ is far lower. This is surprising give that when deployed on the phantom, this protocol delivered accurate results over the range of $T_2$ we see in the kidneys. This disparity is due to the additional confounding factors of diffusion and flow that are present in the body. These factors do not affect the \ac{GraSE} sequence to the same degree as the \ac{SE}-\ac{EPI} sequence.

\section{Discussion}
Of the methods explored, the \ac{GraSE} sequence produced the most accurate results on the phantom and superior image quality in-vivo, we will use this sequence in $T_2$ mapping going forward.

\section{Conclusion}

\section{Acknowledgements}

We are grateful for access to the University of Nottingham's Augusta high performance computing service.

\newpage
\section{References}
\defbibheading{bibliography}[\refname]{}
\printbibliography