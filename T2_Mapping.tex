\chapter{Assessment of Renal $T_2$ Mapping Methods}
\label{chap:t2_mapping}

\begin{abstract}
	Renal \ttwo mapping shows promising early results for the evaluation of multiple pathologies, however, there is very little consistency between studies with different methodologies being employed by each research group. Here we evaluate a basic spin echo, multi echo-spin echo, gradient spin echo method and \ac{CPMG} \ttwo prep method, the four most common \ttwo mapping sequences for use in the kidneys.
	
	Each of the four sequences was used to image a phantom with an array of spheres of known \ttwo to evaluate quantitative accuracy across the range of \ttwo reported in the kidneys. The sensitivity of each sequence to flow was evaluated using a different phantom over a range of flow rates. Additionally, the image quality of each sequence was assessed by estimating the point spread function. All sequences were then used generate \ttwo maps of five healthy volunteers. 
	
	The \ac{CPMG} \ttwo prep sequence delivered the most accurate quantitative results over the range of \ttwo within the static phantom, however, its sensitivity to flow and wide point spread function limit its use in-vivo. Instead, a gradient spin echo sequence is recommended, with a mean relative error of 15 ± 4 \% over the range of \ttwo reported within the kidneys (40 ms – 200 ms), superior readability due to its smaller point spread function and insensitivity to flow.
	
	This work was presented as an aural presentation at the \ac{ISMRM} 28th Annual Meeting (2020) \cite{daniel_comparison_2020}.
	
%	\lipsum[1]
\end{abstract}
\newpage
\acresetall

\section{Introduction}
\label{sec:t2_intro}
Quantitative \ac{MRI} is the process of taking measurements where the voxel values have numerical significance rather than simply representing signal intensity in arbitrary units \cite{tofts_quantitative_2003}. These numerically significant values can take the form of macroscale properties such as rate of oxygen consumption and blood vessel flow rates or microscale properties such as tissue \tone and susceptibility. When interpreted these values can be used improve diagnostic and treatment of patients.

The kidneys are structurally and functionally complex organs and as such lend themselves to the wide variety of MRI protocols designed to probe different aspects of the tissue and processes carried out within. While high resolution images of the kidneys morphology and basic measures such as \ac{TKV} can be very useful in diagnosing and monitoring disease progression \cite{buchanan_quantitative_2019, chapman_kidney_2012, gong_relationship_2012}, these do not fully leverage the quantitative nature of \ac{MRI}. Measurements of \tone have been shown to correlate well with fibrosis in the myocardium \cite{bull_human_2013, ferreira_t1_2013}, liver \cite{hoad_study_2015, luetkens_quantification_2018} and kidneys \cite{friedli_new_2016} and more generally an increase in \tone is associated with \ac{CKD} \cite{gillis_non-contrast_2016, cox_multiparametric_2017, buchanan_quantitative_2019}. \ac{ASL} techniques can be used to quantify renal perfusion in physiological units (mL/100g/min) and has been shown to be correlated with allograft function post renal transplant in addition to cold ischemia time and the recipients \ac{eGFR} \cite{hueper_functional_2015, artz_arterial_2011, ren_evaluation_2016, niles_longitudinal_2016}. Additionally \ac{ASL} has been used to measure a decrease in perfusion in \ac{CKD} subjects \cite{gillis_non-contrast_2016, rossi_histogram_2012, tan_renal_2014}. These techniques have proved useful when used individually however they can be combined and used in the same scanning session to greater effect as a multiparametric protocol \cite{buchanan_quantitative_2019, cox_multiparametric_2017, eckerbom_multiparametric_2019, schley_multiparametric_2018, hueper_kidney_2016}.

\ttwo mapping has found wide use in cardiac \ac{MRI} for assessment of myocardial edema \cite{gouya_rapidly_2008, giri_t2_2009, nasenstein_cardiac_2014} and iron overload \cite{guo_myocardial_2009, krittayaphong_detection_2017}. It has also effectively been used in the brain to study multiple sclerosis \cite{neema_t1-_2007}, epilepsy \cite{rugg-gunn_whole-brain_2005}, dementia \cite{knight_quantitative_2016} and Parkinson’s disease \cite{vymazal_t1_1999}. Despite these developments elsewhere in the body, \ttwo mapping has had limited uptake in the renal community.

Renal \ttwo mapping has seen most research focusing on repeatability measures \cite{de_bazelaire_mr_2004, zhang_reproducibility_2011, li_measuring_2015, de_boer_multiparametric_2020} with clinical uses in the field of assessment of allograft function in mice \cite{hueper_kidney_2016} and humans \cite{mathys_t2_2011, adams_multiparametric_2020}, and has shown potential for early diagnosis of \ac{ADPKD} \cite{franke_magnetic_2017} and assessment of clear cell renal cell carcinoma \cite{adams_use_2019}.

In the existing literature, there is a substantial variation in quoted \ttwo values for the kidneys of healthy volunteers, this is thought to be, in part, due to the differences in \ttwo mapping methodologies. There are currently four main methods, a basic spin echo method, a multi echo-spin echo method, a gradient spin echo method and a \ac{CPMG} \ttwo prep method. Here we aim to compare each of these methods in the context of renal \ttwo mapping to ascertain which is most suitable. This involves evaluating each methods quantitative accuracy, image quality, susceptibility to flow and suitability for use in-vivo.

\section{Methods}
\label{sec:t2_methods}

\subsection{Data Acquisition}
\label{subsec:t2_acq_schemes}

All data was acquired on a 3T Philips Ingenia system (Philips Medical Systems, Best, The Netherlands). The 14 element \ttwo array of a QalibreMD System Standard Model 130 containing spheres doped with varying concentrations of \ce{MnCl_2} to modulate \ttwo between 5 ms and 650 ms was used to compare the accuracy of \ttwo measurements to a known ground truth, Figure \ref{fig:t2_phantom_schematic}. Additionally, a square grid etched into plate three of the phantom was used to assess the degree of image blurring. 

\begin{figure}[H]
	\centering
	\includegraphics[width=0.5\textwidth]{T2_Mapping/Phantom_Example/T2_Spheres.eps}
	\caption{A schematic of the $T_2$ spheres in the QalibreMD phantom.}
	\label{fig:t2_phantom_schematic}	
\end{figure}

To investigate the effects of flow upon \ttwo measurements, a Gold Standard Phantoms \ac{QASPER} phantom was used. This allows the \ttwo of the perfusate to be measured at rest and whilst being pumped through the phantom at a range of flow rates. 

Both phantoms were scanned using a 32-channel head coil. All data acquired on human subjects was done with approval of the local ethics committee and the study was conducted in accordance with the Helsinki Declaration. The subjects gave written, informed consent. Humans were scanned using a 16-channel anterior coil array and 16-channel posterior coil array. The study cohort consisted of 5 healthy participants (2 female, 3 male, mean age 31 $\pm$ 8).

The protocol consisted of a survey, localisers, $B_0$ and $B_1$ mapping, then each of the optimised \ttwo mapping sequences. In-vivo subjects also had \ttwo-weighted and \tone-weighted structural scans to enable segmentation of the whole kidneys, and cortex/medulla respectively \cite{petzold_building_2014, will_automated_2014}, these \ac{ROI} are then used to calculate the mean \ttwo of each tissue type. A summary of the parameters of each \ttwo mapping sequence is shown in Table \ref{tab:t2_sequence_overview}. Each protocol was designed to be approximately two minutes (before respiratory triggering) and keep key parameters such as voxel size and \ac{FOV} constant.

\begin{table}[H]
	\centering
	\begin{adjustbox}{width=1.0\textwidth, center}
	\begin{tabularx}{1.25\textwidth}{X|X|X|X|X}
		                                                     & Spin Echo - Echo Planar Imaging & Multi-Echo Turbo Spin Echo & Gradient Spin Echo & CPMG \ttwo Prep EPI \\ \hline
		Abbreviation                                         & SE-EPI                          & ME-TSE                     & GraSE              & CPMG \ttwo Prep     \\ \hline
		TE (min:step:max) (ms)                               & 20:10:70                        & 13:13:130                  & 11.2:5.6:173.3     & 0:20:160            \\ \hline
		Number of   echoes                                   & 6                               & 10                         & 30                 & 9                   \\ \hline
		Startup echoes                                       & N/A                             & 0                          & 1                  & N/A                 \\ \hline
		TR (ms)                                              & 5000                            & 3000                       & 3000               & 3000                \\ \hline
		Voxel Size   (mm$^3$)                                & 3 $\times$ 3 $\times$ 5         & 3 $\times$ 3 $\times$ 5    &3 $\times$ 3 $\times$ 5 &3 $\times$ 5.65 $\times$ 5\\ \hline
		FoV (mm$^3$)                                         & 288 $\times$ 288 $\times$ 25    & 288 $\times$ 288 $\times$ 25&288 $\times$ 288 $\times$ 25&288 $\times$ 288 $\times$ 25\\ \hline
		Signal   Averages                                    & 2                               & 1                          & 1                  & 1                   \\ \hline
		Acquisition Mode                                     & Multi Slice                     & Multi Slice                & Multi Slice        & Multiple 2D         \\ \hline
		Fast   Imaging Mode                                  & EPI                             & TSE                        & GraSE              & TFEPI               \\ \hline
		Flip Angle                                           & 90$\degree$                     & 90$\degree$                & 90$\degree$        & 90$\degree$         \\ \hline
		Bandwidth   (Hz)                                     & 40                              & 180                        & 405                & 113                 \\ \hline
		SENSE                                                & 2.55                            & 2.55                       & 2.55               & 3                   \\ \hline
		Halfscan                                             & 0.838                           & No                         & No                 & 0.706               \\ \hline
		TSE Factor                                           & N/A                             & 10                         & 30                 & N/A                 \\ \hline
		EPI Factor                                           & 37                              & N/A                        & 3                  & 17                  \\ \hline
		Respiratory Compensation                             & Triggered                       & Triggered                  & Triggered          & Triggered           \\ \hline
		Acquisition Time \scriptsize{(before respiratory compensation)} & 3 min 0 sec                & 1 min 57 sec               & 2 min 6 sec        & 2 min 23 sec  
	\end{tabularx}
	\end{adjustbox}
	\caption{A summary of the acquisition parameters of each of the \ttwo mapping methods compared.}
	\label{tab:t2_sequence_overview}
\end{table}

\subsubsection{Spin Echo-Echo Planar Imaging}
This technique is the simplest of the four consisting of a 90$\degree$ excitation pulse, followed by a 180$\degree$ \ac{RF} pulse \ac{TE}/2 ms later, leading to an echo at \ac{TE}. This 180$\degree$ pulse corrects for components of the signal lost due to static field inhomogeneities however does not correct for \ttwo effects therefore by repeating the sequence multiple times with different \ac{TE}, the \ttwo decay can be sampled. An \ac{EPI} readout is used to sample the signal during the echo. This sequence suffers from a relatively low \ac{SNR}, hence the two signal averages; this means that only six different echo times are recorded. An overview of the sequence is shown in Figure \ref{fig:t2_se-epi_seq}.

\begin{figure}[H]
	\centering
	\includegraphics[width=0.7\textwidth]{T2_Mapping/Pulse_Diagrams/T2_SE_EPI.eps}
	\caption{A pulse sequence diagram of the \ac{SE}-\ac{EPI} scheme.}
	\label{fig:t2_se-epi_seq}	
\end{figure}

\subsubsection{Multi-Echo Turbo Spin Echo}
This sequence also uses a multi-slice spin echo; however the \ac{EPI} readout is replaced by a \ac{TSE} readout. This replaces the singe 180$\degree$ pulse with a train of pulses with an echo forming between each allowing the whole \ttwo decay to be sampled in a single echo train. Different echo times are sampled by varying the number and spacing of the 180$\degree$ pulses. The decrease in acquisition time per echo compared to the \ac{SE}-\ac{EPI} sequence meant that ten echoes were collected per \ttwo map at the minimum possible echo spacing, 13 ms. A schematic of the \ac{PSD} can be seen in Figure \ref{fig:t2_me-tse_seq}.

\begin{figure}[H]
	\centering
	\includegraphics[width=0.7\textwidth]{T2_Mapping/Pulse_Diagrams/T2_ME_TSE.eps}
	\caption{A pulse sequence diagram of the \ac{ME-TSE} scheme.}
	\label{fig:t2_me-tse_seq}	
\end{figure}

\subsubsection{Gradient Spin Echo}

To achieve further acceleration over the \ac{ME-TSE} sequence, a \ac{GraSE} sequence can be used. Here two gradient echoes are collected for every spin echo with the spin echo and gradient echoes being used for the acquisition of the centre and periphery of k-space respectively. The multiple $k$-space profiles collected per spin echo enables a decrease in the echo spacing of the sequence compared to the \ac{ME-TSE}, decreasing to 5.6 ms and thus thirty echoes are collected per \ttwo map. The \ac{PSD} for the \ac{GraSE} acquisition is shown in Figure \ref{fig:t2_grase_seq}.

\begin{figure}[H]
	\centering
	\includegraphics[width=0.7\textwidth]{T2_Mapping/Pulse_Diagrams/T2_GraSE.eps}
	\caption{A pulse sequence diagram of the \ac{GraSE} scheme.}
	\label{fig:t2_grase_seq}	
\end{figure}

\subsubsection{CPMG $T_2$ Preparation}

The \ac{CPMG} \ttwo preparation sequence is considered the gold standard in terms of accuracy. This sequence consists of the 90$\degree$ excitation pulse to transfer the magnetisation into the transverse plane as with the other sequences. A series of spatially non-selective 180$\degree$ pulses with alternating phases are then applied; by varying the number and temporal spacing, $\tau_{\textup{CPMG}}$, of these pulses the degree of \ttwo weighting can me modulated to achieve different \ac{eTE}. An \ac{EPI} readout scheme is then used to sample the signal. An overview of this sequence is shown in Figure \ref{fig:t2_cpmg_t2prep_seq}.

\begin{figure}[H]
	\centering
	\includegraphics[width=0.7\textwidth]{T2_Mapping/Pulse_Diagrams/T2_T2_Prep.eps}
	\caption{A pulse sequence diagram of the \ac{CPMG} $T_2$ preperation scheme.}
	\label{fig:t2_cpmg_t2prep_seq}	
\end{figure}


\subsection{Post Processing}

All post processing was performed using Python 3.7 making use of the \ac{UKAT} toolbox \cite{nery_ukrin_2020}. All curve fitting uses a least squares trust region reflective method to estimate the variables in the desired function.

\subsubsection{\ttwo Fitting Methods}

The data were fit using a mono-exponential model, however, there are multiple models for fitting noisy data, each of which was evaluated here and illustrated in Figure \ref{fig:t2_fitting_methods} The basic fit simply takes the signal from each voxel at each \ac{TE} and fits it to 
\begin{equation}
	S(t) = S_0 \cdot e^{-t/T_2}.
	\label{eq:t2}
\end{equation}
If no noise were present in the data, this would be the optimum method however, decreased SNR of later \ac{TE} often leads to inaccurate fits. To combat this, the equation the data is fit to can be modified to
\begin{equation}
	S(t) = S_0 \cdot e^{-t/T_2} + \epsilon
	\label{eq:t2_noise}
\end{equation}
where $\epsilon$ represents thermal noise and a baseline in the signal due to long \ttwo compartments, this fitting method is referred to as ``noise fit''. Another common method of negating the effects of the low \ac{SNR} later \ac{TE} is to discard data below a threshold, illustrated here at 0.2 AU and referred to as ``discard fit''. Finally, a combination of both noise estimation and discarding was performed (discard and noise fit).

\begin{figure}[H]
	\centering
	\includegraphics[width=0.7\textwidth]{T2_Mapping/fitting_methods.pdf}
	\caption{Each of the four fitting methods being used to estimate \ttwo of the same simulated data.}
	\label{fig:t2_fitting_methods}	
\end{figure}

These fitting methods can either be applied on a voxel-by-voxel basis to generate spatial maps or the signal from all voxels in an \ac{ROI} with a single \ttwo can be averaged at each \ac{TE} with parameters being fit to the subsequent mean signal. 

\subsection{Assessment of Data}

\subsubsection{Quantifying Accuracy}
Using the QalibreMD phantom, the quantitative accuracy of each of the sequences was assessed. A \ac{ROI} was defined for each of the spheres in the \ttwo array and the mean of all voxels within the sphere at each \ac{TE} calculated. The estimated values of \ttwo were compared to the ground truth literature values and their discrepancy assessed over both the full range of \ttwo in the array (5 ms – 650 ms) and the range of \ttwo reported in the kidneys (40 ms – 200 ms) \cite{wolf_magnetic_2018}. Accuracy was summarised over these ranges using \ac{MRE} defined as 
\begin{equation}
	\textup{MRE} = \frac{1}{N}\sum_{n}^{i=1}\left|  \frac{t_{2\; i}^{\textup{ground truth}} - t_{2\; i}^{\textup{estimate}}}{t_{2\; i}^{\textup{ground truth}}}\right|.
	\label{eq:t2_mre}
\end{equation}

\subsubsection{Effects of Flow}
The kidneys are highly perfused organ, as such, the effects of fluid flow through the area being imaged should be evaluated. This was achieved using a Gold Standard Phantoms \ac{QASPER} phantom, Figure \ref{fig:t2_flow_phantom_schematic}. This phantom comprises of a \ac{MRI} compatible pump with adjustable continuous flow rates from 0 m$\ell$/min to 350 m$\ell$/min, the perfusate exiting this pump continues into a series of simulated arterioles before entering a porous media designed to simulate a capillary bed. The porous media is imaged using each sequence over the full range of flow rates the phantoms pump can deliver. \ttwo maps are calculated and the mean \ttwo calculated to evaluate the robustness of each sequence to variations in perfusion.

\begin{figure}[H]
	\centering
	\includegraphics[width=0.5\textwidth]{T2_Mapping/Flow/Flow_schematic.eps}
	%	\missingfigure{Flow phantom schematic}
	\caption{A schematic of \ac{QASPER} phantom used to quantify the effects of flow upon the \ttwo measurements.}
	\label{fig:t2_flow_phantom_schematic}	
\end{figure}

\subsubsection{Blurring}
Unfortunately \ac{MRI} doesn't produce perfect images, every signal is subject to a degree of blurring or spreading out into surrounding voxels. The amount of this blurring is different for each sequence and can dramatically effect the readability of an image and ultimately its clinical utility. In \ac{MRI} the amount and characteristics of the blur are usually spatially invariant, that is to say, if a voxel in the centre of an image is blurred over its five neighbouring voxels in the phase encode direction, a voxel at the edge of the image would have the same five voxel blur applied to it. We wish to quantify the amount of blurring produced by each of the sequences outlined in \ref{subsec:t2_acq_schemes}.

The observed image, $h$, can be modelled as a the ideal, unblurred signal, $f$ distorted by a filter, $g$, Figure \ref{fig:t2_1d_blur}. This distorting filter is known as the \ac{PSF} and is the theoretical signal produced by an infinitely small point source object or, in practice, the blurring observed in the imaging system produced when an object much smaller than the systems resolving power is imaged. In a spatially invariant system such as \ac{MRI} the recorded signal is simply a convolution of the true signal and the \ac{PSF} i.e. $f \ast g = h$. By fitting a Gaussian to the \ac{PSF} we can quantify the degree of blurring in the image \cite{chaimow_more_2017, chaimow_more_2017-1}. 
\begin{figure}[H]
	\centering
	\includegraphics[width=0.8\textwidth]{T2_Mapping/1D_Blur/1d_conv.eps}
	\caption{The convolution of the ideal signal, $f$, and the \ac{PSF}, $g$, produces the measured signal, $h$.}
	\label{fig:t2_1d_blur}	
\end{figure}

A grid etched into one of the plastic plates of the QalibreMD phantom was imaged using each of the \ttwo mapping methods with an echo time of 20 ms and a 0.5 mm$^3$ isotropic structural scan collected. The resolution of the structural scan is much greater than the resolution of the \ttwo mapping scans, therefore it can be seen as an approximation of the ideal image produced by each \ttwo mapping method. This allows a deconvolution of the \ac{PSF} from the \ttwo mapping scans and, by fitting a Gaussian to the line profiles of the \ac{PSF} in each direction, estimate the \ac{FWHM} of the \ac{PSF} as shown in Figure \ref{fig:t2_2d_blur}. Quoted values are the maximum \ac{PSF} of each of the two directions as this is the limiting factor in an images readability, for the example in Figure \ref{fig:t2_2d_blur}, the quoted \ac{PSF} would be 9.49 $\pm$ 0.23 mm.

\begin{figure}[H]
	\centering
	\includegraphics[width=0.9\textwidth]{T2_Mapping/2D_Blur/Example/2D_Blur.eps}
	\caption{An overview of the estimation of the \ac{PSF}. The \ttwo weighted data has had additional blur added to make the effect of each processing step clearer.}
	\label{fig:t2_2d_blur}	
\end{figure}

\subsubsection{In-Vivo}

Using the \tone-weighted structural scans \ac{ROI} were defined for both the renal cortex and medulla of each subject. These \ac{ROI} were the applied to \ttwo maps generated using each method and a mean and standard deviation of \ttwo for each tissue type calculated. \ttwo maps were also qualitatively assessed.

\section{Results}

\subsection{Fitting Methods}

Each fitting method was tested on data acquired on the QalibreMD phantom and in-vivo. A full breakdown of the accuracy of each fitting method, applied to each sequence, over different ranges of \ttwo can be seen in Table \ref{tab:t2_phantom_fitting_methods} and Figure \ref{fig:t2_in_vivo_fitting_methods}. The designers of the phantom calculate the mean signal from each sphere then fit to a \ttwo decay, resulting in a quicker and more accurate measurement of the homogenous \ttwo of each sphere \cite{mristandards_mristandardsphantomviewer_2020}. Due to the heterogeneity within the kidneys, this method could not be applied in-vivo and as such, the accuracy of the sequences and fitting methods was evaluated by performing a voxel-by-voxel fit, then calculating the mean \ttwo from an \ac{ROI} of each sphere in the resulting map.

\begin{table}[H]
	\centering
	\begin{adjustbox}{width=1.0\textwidth, center}
	\begin{tabularx}{1.3\textwidth}{X|X|X|X|X|X|X|X|X|X}
		\ttwo   Range            & \multicolumn{4}{c|}{MRE   (5 ms – 650 ms) (\%)}                         & \multicolumn{5}{c}{MRE   (40 ms – 200 ms) (\%)}              \\ \hline
		Fitting   Method      & Basic     & Noise     & Discard   & Discard   and Noise & \multicolumn{2}{l|}{Basic}     & Noise      & Discard   & Discard   and Noise \\ \hline
		SE-EPI                & 36 $\pm$  34   & 202 $\pm$  437 & 33 $\pm$  34   & 238 $\pm$  463           & \multicolumn{2}{l|}{8 $\pm$  5}     & 32 $\pm$  42    & 22 $\pm$  8    & 31 $\pm$  42             \\ \hline
		ME-TSE                & 38   $\pm$  31 & 13   $\pm$  16 & 41   $\pm$  35 & 35   $\pm$  36           & \multicolumn{2}{l|}{23   $\pm$  13} & 14   $\pm$  3   & 15   $\pm$  13 & 13   $\pm$  4            \\ \hline
		GraSE                 & 32 $\pm$  29   & 23 $\pm$  27   & 26 $\pm$  28   & 33 $\pm$  37             & \multicolumn{2}{l|}{15 $\pm$  4}    & 11 $\pm$  5     & 13 $\pm$  4    & 9 $\pm$  7               \\ \hline
		CPMG   \ttwo Prep        & 18   $\pm$  15 & 30   $\pm$  53 & 20   $\pm$  18 & 28   $\pm$  28           & \multicolumn{2}{l|}{11   $\pm$  1}  & 8   $\pm$  5    & 11   $\pm$  1  & 6   $\pm$  5             \\ \hline
		Mean   over sequences & 31   $\pm$  8  & 67   $\pm$  90 & 30   $\pm$  9  & 83   $\pm$  103          & \multicolumn{2}{l|}{14   $\pm$  7}  & 16   $\pm$  11  & 14   $\pm$  36 & 15   $\pm$  11          
	\end{tabularx}
	\end{adjustbox}
	\caption{\ac{MRE} when measuring \ttwo of the QalibreMD phantom over different ranges using each sequence and fitting method. 5 ms – 650 ms is the full range of \ttwo available in the phantom and 40 ms – 200 ms is the range of \ttwo reported in the kidneys.}
	\label{tab:t2_phantom_fitting_methods}
\end{table}

\begin{figure}[H]
	\centering
	\begin{subfigure}[c]{0.47\textwidth}
		\centering
		\includegraphics[width=1\textwidth]{T2_Mapping/Fitting_methods_invivo/basic.eps}
		\caption{Basic fit}
		\label{fig:t2_in_vivo_fitting_methods_basic}
	\end{subfigure}
	\hfill
	\begin{subfigure}[c]{0.47\textwidth}
		\centering
		\includegraphics[width=1\textwidth]{T2_Mapping/Fitting_methods_invivo/noise.eps}
		\caption{Noise fit}
		\label{fig:t2_in_vivo_fitting_methods_noise}
	\end{subfigure}

	\vskip\baselineskip
	\begin{subfigure}[c]{0.47\textwidth}
		\centering
		\includegraphics[width=1\textwidth]{T2_Mapping/Fitting_methods_invivo/discard.eps}
		\caption{Discard fit}
		\label{fig:t2_in_vivo_fitting_method_discards}
	\end{subfigure}
	\hfill
	\begin{subfigure}[c]{0.47\textwidth}
		\centering
		\includegraphics[width=1\textwidth]{T2_Mapping/Fitting_methods_invivo/noise_discard.eps}
		\caption{Discard and noise fit}
		\label{fig:t2_in_vivo_fitting_methods_noise_discard}
	\end{subfigure}
	\caption{\ttwo maps generated using each of the fitting methods using the \ac{GraSE} sequence. The reduction in longer \ttwo values when fit with a noise term observed in the phantom can be seen here in the decreased \ttwo of the kidneys, while the spleen remains relatively similar. Discarding has relatively little effect on the kidneys but does increase the variance in \ttwo within the cortex and medulla compared to the basic fit.}
	\label{fig:t2_in_vivo_fitting_methods}
\end{figure}

The noise fit results in an increase accuracy compared to the basic fit when measuring the \ttwo of the shortest \ttwo spheres, especially for sequences with a short echo spacing where the majority of echoes in the signal are after the sphere has fully relaxed back to its baseline noise level. This increase in accuracy for short \ttwo is at the expense of the accuracy of the long \ttwo spheres. The combination of decreased dynamic range and an extra parameter to optimise resulted in an inaccurate characterisation of these spheres, especially of the \ac{SE}-\ac{EPI} sequence due to the short final echo time and thus lower dynamic range. In the range of \ttwo reported in the kidney, the noise fit increased the average error over the four sequences. 

The discard fit requires an empirical threshold to be chosen. If chosen correctly, this resulted in slightly improved accuracy however if the threshold was not correctly chosen the accuracy of \ttwo was compromised. While optimising the threshold is trivial when the known reference values are available, this is more difficult in-vivo and therefore, given the increase in accuracy was marginal and only effected spheres of short \ttwo, consistent results were deemed preferable.

The combination of discard and noise fit resulted in a decreased accuracy from the basic fit. Spheres that benefited from the additional noise term relax to their baseline noise quickly, however by discarding these echoes, the estimate of noise becomes inaccurate and thus so does the estimate of \ttwo.

It was therefore concluded that a basic fit should be used for all subsequent renal data.

\subsection{Phantom Verification}

\subsubsection{Accuracy}

Each sequence was used to image the T2 array in the QalibreMD phantom, Figure \ref{fig:t2_phantom_loc}. Figure \ref{fig:t2_phantom_cor} shows the measured \ttwo plot against the reference \ttwo for each method. All methods struggle to accurately measure the very short \ttwo spheres in the array with the \ac{CPMG} \ttwo prep method faring best. The \ac{SE}-\ac{EPI} method overestimates short \ttwo spheres due to the first \ac{TE} being sampled at 20 ms but is also underestimating long \ttwo because of its small range in \ac{TE}. When considering only the spheres with physiologically similar \ttwo to the kidneys, the \ac{SE}-\ac{EPI} method is the most accurate with the \ac{GraSE} and \ac{CPMG} \ttwo prep delivering similar results. This is mirrored by the \ac{MRE} shown in Table \ref{tab:t2_phantom_acquisition_methods}. Example \ttwo maps of the phantoms \ttwo array imaged with each sequence are shown in Figure \ref{fig:t2_phantom_maps}.

\begin{figure}[H]
	\centering
	\includegraphics[width=0.5\textwidth]{T2_Mapping/Phantom_Example/Localiser_multi.eps}
	\caption{The $T_2$ spheres inside the QalibreMD phantom.}
	\label{fig:t2_phantom_loc}	
\end{figure}

\begin{figure}[H]
	\centering
	\begin{subfigure}[c]{0.47\textwidth}
		\centering
		\includegraphics[width=1\textwidth]{T2_Mapping/Phantom_cor/basic_fit_full.pdf}
		\caption{}
		\label{fig:t2_phantom_cor_full}
	\end{subfigure}
	\hfill
	\begin{subfigure}[c]{0.47\textwidth}
		\centering
		\includegraphics[width=1\textwidth]{T2_Mapping/Phantom_cor/basic_fit_kidney.pdf}
		\caption{}
		\label{fig:t2_phantom_cor_kidney}
	\end{subfigure}
	\caption{\ttwo measured using each method compared to the reference \ttwo from literature. (\subref{fig:t2_phantom_cor_full}) The full range of \ttwo spheres is shown on logarithmic axis with the range of \ttwo reported in the kidneys shaded in red (\subref{fig:t2_phantom_cor_kidney}) The spheres with \ttwo in the range of the kidneys are shown on linear axis.}
	\label{fig:t2_phantom_cor}
\end{figure}

\begin{table}[H]
	\centering
	\begin{tabularx}{1\textwidth}{X|X|X}
		Acquisition   Method & MRE   (5 ms – 650 ms) (\%) & MRE   (40 ms – 200 ms) (\%) \\ \hline
		SE-EPI               & 36 $\pm$ 34                    & 8   $\pm$ 5                     \\ \hline
		ME-TSE               & 38   $\pm$ 31                  & 23 $\pm$ 13                     \\ \hline
		GraSE                & 32 $\pm$ 29                    & 15   $\pm$ 4                    \\ \hline
		CPMG   \ttwo Prep       & 18   $\pm$ 15                  & 11   $\pm$ 1                   
	\end{tabularx}
	\caption{\ac{MRE} when measuring \ttwo of the QalibreMD phantom over different ranges using each sequence. 5 ms – 650 ms is the full range of \ttwo available in the phantom and 40 ms – 200 ms is the range of \ttwo expected in the kidneys.}
	\label{tab:t2_phantom_acquisition_methods}
\end{table}

\begin{figure}[H]
	\centering
	\begin{subfigure}[c]{0.47\textwidth}
		\centering
		\includegraphics[width=1\textwidth]{T2_Mapping/Phantom_maps/SE.pdf}
		\caption{\ac{SE}-\ac{EPI}}
		\label{fig:t2_phantom_map_se}
	\end{subfigure}
	\hfill
	\begin{subfigure}[c]{0.47\textwidth}
		\centering
		\includegraphics[width=1\textwidth]{T2_Mapping/Phantom_maps/ME.pdf}
		\caption{\ac{ME-TSE}}
		\label{fig:t2_phantom_map_me_tse}
	\end{subfigure}
	
	\vskip\baselineskip
	\begin{subfigure}[c]{0.47\textwidth}
		\centering
		\includegraphics[width=1\textwidth]{T2_Mapping/Phantom_maps/GraSE.pdf}
		\caption{\ac{GraSE}}
		\label{fig:t2_phantom_map_grase}
	\end{subfigure}
	\hfill
	\begin{subfigure}[c]{0.47\textwidth}
		\centering
		\includegraphics[width=1\textwidth]{T2_Mapping/Phantom_maps/CPMG.pdf}
		\caption{\ac{CPMG} \ttwo Prep}
		\label{fig:t2_phantom_map_cpmg_t2_prep}
	\end{subfigure}
	\caption{\ttwo maps of the QaliberMD system phantom \ttwo array generated using each sequence.}
	\label{fig:t2_phantom_maps}
\end{figure}

\subsubsection{Sensitivity to Flow}

The simulated capillary bed of the flow phantom was imaged with the perfusate being pumped at rates from 0 to 350 m$\ell$/min. The absolute change in measured \ttwo is shown in Figure \ref{fig:t2_flow_abs} and the change in \ttwo as a percentage of \ttwo measured when the pump was turned off is shown in Figure \ref{fig:t2_flow_percent}. The \ttwo of the perfusate is relatively long and as such the inaccuracies measuring long \ttwo observed in the static phantom (Figure \ref{fig:t2_phantom_cor_full}) manifest themselves here. This causes a large range in \ttwo even when the pump is turned off. In Figure 5b the \ac{SE}-\ac{EPI} sequence can be seen to be most sensitive to perfusion due to its largest deviation as flow rate increased, the \ac{GraSE} sequence produced the minimum proportional change in \ttwo as the rate perfusate was pumped through the capillary bed was increased.
\begin{figure}[H]
	\centering
	\begin{subfigure}[c]{0.47\textwidth}
		\centering
		\includegraphics[width=1\textwidth]{T2_Mapping/Flow/flow_abs.pdf}
		\caption{}
		\label{fig:t2_flow_abs}
	\end{subfigure}
	\hfill
	\begin{subfigure}[c]{0.47\textwidth}
		\centering
		\includegraphics[width=1\textwidth]{T2_Mapping/Flow/flow_per.pdf}
		\caption{}
		\label{fig:t2_flow_percent}
	\end{subfigure}
	\caption{The effects of flow on measurements of \ttwo using each method. Points have been slightly staggered on the $x$-axis to make the error bars visible. (\subref{fig:t2_flow_abs}) Absolute change in measured \ttwo. (\subref{fig:t2_flow_percent}) Change in \ttwo as a percentage of \ttwo measured at rest for each sequence.}
	\label{fig:t2_flow}
\end{figure}

The flow rate is measured at the pump, therefore the perfusate will not be travelling at the reported rate through the capillary bed. To quantify the movement of the perfusate using techniques widely available in the kidneys, \ac{ADC} maps of the capillary bed were calculated at each flow rate, Figure \ref{fig:t2_flow_adc}.

\begin{figure}[H]
	\centering
	\includegraphics[width=0.6\textwidth]{T2_Mapping/Flow/flow_adc.pdf}
	\caption{Changes in ADC as the rate perfusate is pumped through the simulated capillary bed is increased.}
	\label{fig:t2_flow_adc}	
\end{figure} 

\subsubsection{Image Quality}

Each sequence was slightly modified to contain a volume with TE 20 ms and used to image the orthogonal grid in the QalibreMD phantom. A high-resolution structural scan was then used to deconvolve an estimate of the \ac{PSF} from each of the \ttwo-weighted images, Table \ref{tab:t2_phantom_blur}. 

\begin{table}[H]
	\centering
	\begin{tabular}{l|l}
		Acquisition   Method & Point   Spread Function FWHM (mm) \\ \hline
		SE-EPI               & 4.80   $\pm$ 0.18                 \\ \hline
		ME-TSE               & 4.20 $\pm$ 0.14                   \\ \hline
		GraSE                & 4.26   $\pm$ 0.12                 \\ \hline
		CPMG   \ttwo Prep    & 6.48   $\pm$ 0.33                    
	\end{tabular}
	\caption{The \ac{FWHM} of the estimated \ac{PSF} for each acquisition method.}
	\label{tab:t2_phantom_blur}
\end{table}

Both the \ac{SE}-\ac{EPI} and \ac{CPMG} \ttwo prep sequences suffer from significant image distortions due to their \ac{EPI} readout with the \ac{ME-TSE} and \ac{GraSE} both producing similar image quality and a comparable width of \ac{PSF}. Examples of the central slice of the grid imaged with each method are shown in Figure \ref{fig:t2_phantom_blur}. 

\begin{figure}[H]
	\centering
	\begin{subfigure}[c]{0.30\textwidth}
		\centering
		\includegraphics[width=1\textwidth]{T2_Mapping/Phantom_blur/gold.png}
		\caption{Structural Scan}
		\label{fig:t2_phantom_blur_gold}
	\end{subfigure}
	\hfill
	\begin{subfigure}[c]{0.30\textwidth}
		\centering
		\includegraphics[width=1\textwidth]{T2_Mapping/Phantom_blur/se.png}
		\caption{\ac{SE}-\ac{EPI}}
		\label{fig:t2_phantom_blur_se}
	\end{subfigure}
	\hfill
	\begin{subfigure}[c]{0.30\textwidth}
		\centering
		\includegraphics[width=1\textwidth]{T2_Mapping/Phantom_blur/me.png}
		\caption{\ac{ME-TSE}}
		\label{fig:t2_phantom_blur_me}
	\end{subfigure}
	\vskip\baselineskip
%	\hbox to 0.3\textwidth{}% !!
	\hfill
	\begin{subfigure}[c]{0.30\textwidth}
		\centering
		\includegraphics[width=1\textwidth]{T2_Mapping/Phantom_blur/grase.png}
		\caption{\ac{GraSE}}
		\label{fig:t2_phantom_blur_grase}
	\end{subfigure}
%	\hfill
	\hbox to 6.3mm{}% !!
	\begin{subfigure}[c]{0.30\textwidth}
		\centering
		\includegraphics[width=1\textwidth]{T2_Mapping/Phantom_blur/cpmg.png}
		\caption{\ac{CPMG} \ttwo Prep}
		\label{fig:t2_phantom_blur_cpmg}
	\end{subfigure}
%	\hfill
	\caption{Examples of the orthogonal grid imaged with a \ac{TE} of 20 ms using each sequence and the high-resolution structural scan used as the gold standard to be deconvolved from each \ttwo-weighted scan.}
	\label{fig:t2_phantom_blur}
\end{figure}

\subsection{In-Vivo}

Five healthy volunteers were imaged using each of the four methods. Example in-vivo images are shown in Figure \ref{fig:t2_in_vivo_maps}. The \ttwo measured using the \ac{SE}-\ac{EPI} method is much lower than when measured using each of the other three methods. The \ac{GraSE} shows the largest contrast between cortex and medulla while the \ac{CPMG} \ttwo prep suffers from image artefacts with the large amount of blurring measured on the phantom making the in-vivo data difficult to interpret.

\begin{figure}[H]
	\centering
	\begin{subfigure}[c]{0.47\textwidth}
		\centering
		\includegraphics[width=1\textwidth]{T2_Mapping/Invivo/se.eps}
		\caption{\ac{SE}-\ac{EPI}}
		\label{fig:t2_in_vivo_map_se}
	\end{subfigure}
	\hfill
	\begin{subfigure}[c]{0.47\textwidth}
		\centering
		\includegraphics[width=1\textwidth]{T2_Mapping/Invivo/me.eps}
		\caption{\ac{ME-TSE}}
		\label{fig:t2_in_vivo_map_me_tse}
	\end{subfigure}
	
	\vskip\baselineskip
	\begin{subfigure}[c]{0.47\textwidth}
		\centering
		\includegraphics[width=1\textwidth]{T2_Mapping/Invivo/grase.eps}
		\caption{\ac{GraSE}}
		\label{fig:t2_in_vivo_map_grase}
	\end{subfigure}
	\hfill
	\begin{subfigure}[c]{0.47\textwidth}
		\centering
		\includegraphics[width=1\textwidth]{T2_Mapping/Invivo/cpmg.eps}
		\caption{\ac{CPMG} \ttwo Prep}
		\label{fig:t2_in_vivo_map_cpmg_t2_prep}
	\end{subfigure}
	\caption{Example in-vivo \ttwo maps produced using each method.}
	\label{fig:t2_in_vivo_maps}
\end{figure}

\ac{ROI} for the cortex and medulla were defined from the \tone-weighted structural scan and these \ac{ROI} used to calculate the mean \ttwo for each tissue using each mapping method. The mean across the five subjects is shown in Figure \ref{fig:t2_in_vivo_bar}.

\begin{figure}[H]
	\centering
	\includegraphics[width=0.8\textwidth]{T2_Mapping/Invivo/bar.pdf}
	\caption{Mean cortical and medullary \ttwo values across five subjects measured using each sequence.}
	\label{fig:t2_in_vivo_bar}	
\end{figure}

\section{Discussion}

In this study, four of the most common \ttwo mapping methods were optimised for use in the kidneys. These sequences were validated for quantitative accuracy, image quality and sensitivity to flow using phantoms before being used to image five healthy volunteers. Additionally, four different methods of fitting \ttwo maps were compared.

Of the fitting methods compared, the basic fit was deemed the most appropriate. While in some limited situations other methods could produce more accurate results, over the range of \ttwo within the kidneys, fitting for a baseline noise term reduced the calculated \ttwo of longer \ttwo tissues because they had not fully recovered before the final \ac{TE} and as such the noise term was over-estimated. This method would be more appropriate for use in tissues with a shorter \ttwo such as the liver. Discarding \ac{TE} with signal below an empirically derived threshold can result in an increased accuracy for short \ttwo tissues but does not improve the accuracy of the kidneys. The discard threshold can be manually defined; however, this leads to potentially inaccurate results if a sub-optimum threshold is chosen, these inaccurate results can be difficult to identify in-vivo with no gold standard to compare with. Alternatively, an \ac{ROI} over the liver can be defined and the signal from the final \ac{TE} used to inform the estimation of the discard threshold as in most cases, due to its shorter \ttwo, the liver will have fully recovered by the final \ac{TE}. While this method does eliminate the manual aspect of threshold definition between subjects, it requires the delimitation of an additional \ac{ROI} for no increase inaccuracy within the kidneys. It was therefore concluded that the basic fit should be used for renal studies, however, the noise fit should be considered for liver \ttwo measurements.

The only sequence that was able to accurately estimate the \ttwo of the shortest \ttwo spheres of the QalibreMD phantom was the \ac{CPMG} \ttwo prep, Figure \ref{fig:t2_phantom_cor} and Table \ref{tab:t2_phantom_acquisition_methods}. This is because, its able to achieve an effective echo time of 0 ms and thus sample the signal before it has decayed; for other sequences the first echo time can be multiple times the \ttwo of these short spheres and as such the signal has already decayed a large amount. Additionally, the \ac{CPMG} sequence is self-correcting for imperfect 180$\degree$ pulses, and thus the longer \ac{TE} are more accurate. It is the very larger relative error of the shortest \ttwo spheres that leads to the unacceptable large errors in the other sequences over the full range of \ttwo within the phantom, Figure SXa\todo{add figure}. Both the \ac{ME-TSE} and \ac{GraSE} yield similar accuracies over the range of \ttwo seen in the kidneys due to their rapid acquisition, initial \ac{TE} much shorter than the \ttwo of the kidneys and, in the case of \ac{GraSE}, short echo spacing  and thus more \ac{TE} to fit each voxel to. The \ac{SE}-\ac{EPI} sequence suffers due to being relatively slow, this is because the thick slice profile of the 180$\degree$ pulse means that each \ac{TE} needs to be collected in two packets with slices interleaved. As such only a limited number of \ac{TE} can be acquired in a reasonable timeframe and, without any of the more advance elements of the \ac{CPMG} sequence its accuracy is compromised. 

From Figure \ref{fig:t2_flow} it can be observed that the \ac{SE}-\ac{EPI} sequence is most sensitive to flow with the largest proportional change in measured \ttwo as the rate of flow is increased.  The sequences with the lowest absolute \ttwo with the pump off are also the sequences that are influenced most when the pump is turned on, this is because there are still effects of diffusion/flow even with the pump turned off and as such the \ttwo is still reduced. The \ac{ADC} increases from $1.15\times 10^{-3}$ $\pm$ $0.06\times 10^{-3}$ mm$^2$/s to $1.36\times 10^{-3}$ $\pm$ $0.09\times 10^{-3}$ mm$^2$/s over the full range of flow rates the pump can produce, Figure \ref{fig:t2_flow_adc}, corresponding to an increase of 17\% therefore we still expect considerable diffusion effects when the pump is turned off. The \ac{GraSE} sequence is least sensitive to flow with only a 5\% decrease in \ttwo. This is due to the \ac{GraSE} sequences short echo spacing and the fact multiple \ac{TE} are collected within the same \ac{TR}, as such, the effects of diffusion are not consistent across \ac{TE} and do not produce a diffusion weighting proportional to the \ttwo weighting.

The large amount of blurring of the \ac{CPMG} \ttwo prep sequence limits its readability in-vivo and is due to the high \ac{EPI} factor leading to an increased acquisition voxel size. The degree of blurring measured in the other three sequences was comparable with the \ac{SE}-\ac{EPI} having a slightly broader \ac{FWHM}.

Comparing the in-vivo data in Figure \ref{fig:t2_in_vivo_maps} and Figure \ref{fig:t2_in_vivo_bar} we can see the relative strengths and weaknesses of each sequence observed in the phantoms are mirrored in-vivo. The \ac{SE}-\ac{EPI} sequence is measuring much shorter \ttwo than the other three and measures a shorted \ttwo in the medulla than the cortex, this was also observed by de Bazelaire et al \cite{de_bazelaire_mr_2004} and is somewhat surprising given the perfusion of the renal cortex is higher than the medulla \cite{buchanan_quantitative_2019, nery_consensus-based_2020} yet in Figure \ref{fig:t2_flow} it was observed that measured \ttwo decreases as flow increases. This is most likely caused by a difference in the direction of flow between the phantom and the medulla. In the flow phantom the perfusate was flowing radially in from the circumference and thus the flow is predominantly in-plain whereas in the kidney the renal vein quickly ascends to the inferior vena cava and is thus out-of-plain. This out-of-plain flow means that for longer \ac{TE}, some protons have travelled out of the profile of the 180$\degree$ refocusing pulse and as such do not contribute to the signal. This effect manifests itself more at longer \ac{TE} and thus the measured \ttwo decreases.

The \ac{ME-TSE} sequence produces reasonable images and broadly similar \ttwo to the \ac{GraSE} and \ac{CPMG} \ttwo prep for the cortex however the \ttwo measured in the medulla is lower therefore little contrast is seen between the cortex and medulla.  The image quality produced by the \ac{GraSE} sequence is similar to that of the \ac{ME-TSE} which is expected given the comparable \ac{PSF} and sequence architecture. The \ac{GraSE} measures higher medullary \ttwo than the \ac{ME-TSE} resulting in greater corticomedullary contrast. The effects of the wide \ac{PSF} can be seen in the image produced by \ac{CPMG} \ttwo Prep with a large amount of blurring decreasing the readability of the image. The mean \ttwo of the cortex and medulla are similar to that measured using the \ac{GraSE} sequence however, due to the inferior image quality, the variance within the \ac{ROI} is much higher.

Having compared the quantitative accuracy, image quality and acquisition time of four common \ttwo mapping sequences on phantoms and in-vivo for use in the, we conclude that the \ac{GraSE} sequence provides the optimum protocol for renal \ttwo mapping. In phantoms, the accuracy was shown to be comparable with the \ac{ME-TSE} sequence and its superior in-vivo image quality and insensitivity to flow lead us to recommend this sequence for further renal studies.

\section{Conclusion}

A \ac{SE}-\ac{EPI}, \ac{ME-TSE}, \ac{GraSE} and \ac{CPMG} \ttwo Prep sequence were used to image phantoms to assess their accuracy when quantifying \ttwo, sensitivity to flow and image quality by estimating the \ac{PSF}. These sequences were then used to acquire \ttwo maps of the kidneys of five healthy volunteers.

The \ac{GraSE} sequence is recommended for future renal studies due to its superior image quality and accuracy within the time constraints.

All acquisition presented here was carried out on a Philips system. Going forward these methods should be evaluated on other vendors to enable a better comparison of results between sites.

\section{Acknowledgements}

We are grateful for access to the University of Nottingham's Augusta high performance computing service.

\newpage
\section{References}
\defbibheading{bibliography}[\refname]{}
\printbibliography