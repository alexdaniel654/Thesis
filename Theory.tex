\chapter{Principles of Nuclear Magnetic Resonance Imaging}
\label{chap:Theory}

\begin{abstract}
	\lipsum[1]
\end{abstract}
\newpage

\section{Source of the NMR Signal}
\label{sec:theory_source_of_nmr}

\subsection{Nuclear Spin}
\label{subsec:theory_nuclear_spin}
The \ac{NMR} signal arises from the interraction between the atomic nucleus and an external magnetic field. These atomic nuclei poses intrinsic properties, mass ($m$), charge ($q$) and spin ($I$). Spin is a quantum mechanical property and as such, can only take values of half integers or integers. Nuclear spin is dictated by the sum of the constituent particles of the nucleus, protons and neutrons, each of which posses their own spin of either $\sfrac{1}{2}$ or $-\sfrac{1}{2}$. The additive nature of nuclear spin means that pairs of neucleons can cancel out leaivng the nucleus with zero net spin, this happens when the nuclei contains and even number of protons and neutrons. If the nuclei contains an odd number of both protons and neutrons, it will have a positive integer nuclear spin whereas if the nuclei has an odd number of protons or neutrons, it will have a half integer spin. 

\subsection{Nuclear Magnetisation}

The spin angular momentum, $\mathbf{J}$ of a nucleus of spin $I$ is given by
\begin{equation}
\left|\mathbf{J}\right| = \hbar \sqrt{I\left(I+1\right)}
\end{equation}
where $\hbar$ is the reduced Plank's constant, $\sfrac{h}{2\pi}$. As the neucleus is charged and rotating, it gives rise to a current and therefore a magnetic moment $\mathbf{\mu}$,
\begin{equation}
\mathbf{\mu}=\gamma \mathbf{J}
\end{equation}
where $\gamma$ is the gyromagnetic ration for the nucleus.

\section{Relaxation to Contrast}

\section{Forming an Image}