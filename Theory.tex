\chapter{Principles of Nuclear Magnetic Resonance Imaging}
\label{chap:Theory}

\begin{abstract}
	\lipsum[1]
\end{abstract}
\newpage

\section{Source of the NMR Signal}
\label{sec:theory_source_of_nmr}

\subsection{Nuclear Spin}
\label{subsec:theory_nuclear_spin}
The \ac{NMR} signal arises from the interraction between the atomic nucleus and an external magnetic field. These atomic nuclei poses intrinsic properties, mass ($m$), charge ($q$) and spin ($I$). Spin is a quantum mechanical property and as such, can only take values of half integers or integers. Nuclear spin is dictated by the sum of the constituent particles of the nucleus, protons and neutrons, each of which posses their own spin of either $\sfrac{1}{2}$ or $-\sfrac{1}{2}$. The additive nature of nuclear spin means that pairs of neucleons can cancel out leaivng the nucleus with zero net spin, this happens when the nuclei contains and even number of protons and neutrons. If the nuclei contains an odd number of both protons and neutrons, it will have a positive integer nuclear spin whereas if the nuclei has an odd number of protons or neutrons, it will have a half integer spin. 

%\subsection{Nuclear Magnetisation}

The spin angular momentum, $\mathbf{J}$ of a nucleus of spin $I$ is given by
\begin{equation}
\left|\mathbf{J}\right| = \hbar \sqrt{I\left(I+1\right)}
\label{eq:theory_angular_momentum}
\end{equation}
where $\hbar$ is the reduced Plank's constant, $\sfrac{h}{2\pi}$. As the neucleus is charged and rotating, it gives rise to a current and therefore a magnetic moment $\mathbf{\mu}$,
\begin{equation}
\mathbf{\mu}=\gamma \mathbf{J}
\label{eq:theory_magnetic_moment}
\end{equation}
where $\gamma$ is the gyromagnetic ration for the nucleus, a constant which depends on the charge and mass of the nucleus. Table \ref{tab:theory_isotope_spin_gmr} shows the gyromagneitc ratio ($\gamma$) and nuclear spin ($I$) of common \ac{NMR} sensetive isotopes \cite{harris_n.m.r._1976, bernstein_handbook_2004}. Due to its relatively high gyromagnetic ratio, compared to other nuclei used for \ac{NMR}, and relativel abundance in the body, \ce{^{1}H}, a single proton, is most commanly used for \ac{MRI}.

\begin{table}[H]
	\centering
	\begin{tabular}{lccc}
		\hline
		Isotope            & Spin & $\gamma$ (MHzT$^{-1}$)                     & Sensitivity Relative to \ce{^{1}H}     \\ \hline
		\ce{^{1}H}         & $\sfrac{1}{2}$  & 42.58                                      & 1                                      \\
		\ce{^{2}H}         & $1$    & 6.54                                       & 0.0097                                 \\
		\ce{^{13}C}        & $\sfrac{1}{2}$  & 10.71                                      & 0.016                                  \\
		\ce{^{19}F}        & $\sfrac{1}{2}$  & 40.05                                      & 0.83                                   \\
		\ce{^{23}Na}       & $\sfrac{3}{2}$  & 11.27                                      & 0.093                                  \\
		\ce{^{31}P}        & $\sfrac{1}{2}$  & 17.25                                      & 0.066                                  \\ \hline
	\end{tabular}
	\caption{Common \ac{NMR} isotopes, their nuclear spin, gyromagnetic ratio and sensitivity, relative to \ce{^{1}H}.}
	\label{tab:theory_isotope_spin_gmr}
\end{table}

\subsection{Application of an External Magnetic Field}
If we conside the hydrogen nuclei in a sample of tissue, the number of possible eigenstates for a nucleus of nuclear spin $I$ is $\left(2I + 1\right)$. This means that for the \ce{^{1}H} nuclei in our sample where $I=\sfrac{1}{2}$, we can observe two possible eigenstates, $\left| + \sfrac{1}{2} \right \rangle$ and $\left| - \sfrac{1}{2} \right \rangle$ often written as $\left|  \uparrow \right \rangle$ and $\left|  \downarrow \right \rangle$. In the absence of an external magnetic field, these states are degenerate as they have the same energy, however, if we move our sample into a static external magnetic field along the $z$-axis, $B_0$ the energy levels separate.\\

The $z$-componnent of the magnetic moment is defined by,
\begin{equation}
\mu_z=\gamma \hbar \mu_I
\label{eq:theory_longitudinal_magnetic_moment}
\end{equation}
where $m_I$ are the possible spin quantum numbers of the nucleus. For our proton system with spin $\sfrac{1}{2}$, $\mu_z$ is given by
\begin{equation}
\mu_z = \pm \frac{1}{2}\gamma\hbar
\end{equation}

\subsection{Precession}
The nuclear magnetic moment always has a transverse component as from equations \eqref{eq:theory_angular_momentum}, \eqref{eq:theory_magnetic_moment} and \eqref{eq:theory_longitudinal_magnetic_moment} we can see $\left| \mathbf{\mu} \right| > \mu_z$, and therefore the magnetic moment cannot exactly align along the $z$-axis.

\section{Relaxation to Contrast}

\section{Forming an Image}

\newpage
\section{References}
\defbibheading{bibliography}[\refname]{}
\printbibliography