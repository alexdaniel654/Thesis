\chapter{Introduction}
%\addcontentsline{toc}{chapter}{Introduction}
\section{Imaging in the Clinic}
From April 2019 to March 2020, the United Kingdom's \ac{NHS} performed 44,884,450 medical imaging procedures, of these 3,811,415 were \ac{MRI} \cite{noauthor_diagnostic_2020}. This technique can be used to produce high resolution volumetric images of the body with exquisite soft tissue contrast. Unlike other modalities, such as \ac{CT} and \ac{PET}, \ac{MRI} uses non-ionising radiation, making it more suitable for longitudinal analysis of patient progression and research involving healthy volunteers.\\

The superior soft tissue contrast of \ac{MRI} compared to \ac{CT} meant it first found widespread clinical adoption in the field of neuroimaging. Here \ac{MRI} has been used for diagnosis of neurological disorders, monitoring treatment progression and research into cognition. Many of the techniques honed in the brain, can be applied to the abdomen, where similar tissue properties can exploit the same techniques; albeit in a somewhat more challenging environment due to respiratory motion and a more inhomogeneous tissue structure. The kidneys are ideally suited to this translation as they have similar tissue properties to the brain and are highly dynamic organs.\\

In addition to the acquisition of basic structural images, \ac{MRI} can be used to collect quantitative information about the tissues being imaged. In this situation the numerical voxel values have physical significance, rather than simply representing signal intensity in arbitrary units \cite{tofts_quantitative_2003}. Using quantitative \ac{MRI} properties such as an organs oxygen consumption \cite{zhang_quantitative_2015}, perfusion \cite{karger_quantitation_2000}, stiffness \cite{mariappan_magnetic_2010} and temperature \cite{yuan_towards_2012} can be measured. Although many quantitative \ac{MRI} techniques have been developed for the the kidneys, there is still many methods where development, translation from the brain or standardisation with the wider renal community would be highly desirable.

\section{Clinical Motivation. Kidneys; wot do?}
%\addcontentsline{toc}{section}{Clinical Motivation}

The kidneys are two bean shaped organs found in the abdomen, just below the rib cage, symmetrical about the spine. They participate in the control of bodily fluids by regulating the balance of electrolytes, excreting waste product of metabolism and excess water from blood into urine \cite{lote_principles_2012}. \\

Kidneys are made up of units called nephrons, each of which contains a renal corpuscle and a tubule. The renal corpuscle itself is made up of a glomerulus, a cluster of capillaries that allow wastes and fluid to pass out of the blood stream, into the Bowman capsule, the second part of the renal corpuscle, larger structures, such as blood cells and proteins remain in the blood. The substances that passed through the glomerulus are moved to the tubules, each of which has blood vessels running alongside; these vessels reabsorb many of the important components of the blood such as the majority of the water, minerals and nutrients. The remaining fluids and waste in the tubules are collected in the ureter and removed from the body \cite{hall_guyton_2015}.\\

Tissue in the kidney is separated into renal cortex, the outer portion of the kidney and renal medulla, the inner portion. The cortex contains the corpuscles with the tubules passing from the cortex to the medulla. Medullary tissue is compartmentalised into renal pyramids. Blood is supplied to the kidney via the renal artery, this branches into smaller vessels until it reaches the glomeruli then flows out via the renal vein.\\

Due to their vital function in the body and the toxins they encounter as they perform their role, the kidneys are susceptible to problems. \ac{CKD} is the progressive destruction of the kidneys and therefore decrease in renal function. More quantitatively, \ac{CKD} can be assessed clinically by \ac{GFR}, the rate at which fluid is filtered through the kidneys, with a value below 60 ml/min/1.73m$^2$ of body surface area being diagnostic or the presence of albumin, the main protein in blood plasma, in the patients urine \cite{stevens_assessing_2006, farrugia_albumin_2010, pruijm_blood_2017}. Common causes of \ac{CKD} are high blood pressure and diabetes as these damage the nephrons with high blood pressure also posing a risk to the blood vessels within the kidney. Renal tissue is highly vascularised and as such, the risks associated with high blood pressure as especially prevalent in the kidneys. An estimated 5–11\% of the global population suffer from \ac{CKD} \cite{coresh_prevalence_2003, de_lusignan_identifying_2005, drey_population-based_2003, amato_prevalence_2005, chadban_prevalence_2003} making it a significant public health concern. Late referral of renal disorders results in an increase in mortality rate and treatment costs \cite{jungers_late_1993, sesso_late_1996, klebe_cost_2007}. Given that in 2013/2014 renal services cost the \ac{NHS} \pounds 586 million \cite{precious_nhs_2015} there are clear health and economic advantages to an early diagnosis and improved treatment of \ac{CKD}. This can either be achieved via directly aiding diagnosis i.e. developing tools used on to assess patients condition and tailor treatment, or via improving understanding of \ac{CKD} leading to a earlier, more accurate diagnosis using existing techniques and thus more personalised medicine.\\

The current methods available to study \ac{CKD} are not ideal for a variety of reasons. Histological samples are the gold standard for studying renal tissue however collecting them is an invasive process and as such they are not suitable for monitoring the progress of a patient's condition on a regular basis. This coupled with the fact that a small sample is not representative of the entirety of both kidneys means that this method  has large drawbacks. Ultrasound can be used to gather structural information about the kidneys non-invasively, however, it suffers from low spatial resolution and the images being difficult to interpret \cite{hansen_ultrasonography_2015}. The most common method of diagnosis is to estimate \ac{GFR} from the creatinine content in a blood sample however this measure does not allow for the individual assessment of each kidney and is an indirect measure of kidney tissue damage.\\

\ac{MRI} is an ideal modality for the study of kidney disease use to its non-ionising, non-invasive and quantitative nature. A current research interest at the \ac{SPMIC} is the use of multi-parametric quantitative renal \ac{MRI} to assess and predict \ac{CKD} \cite{cox_multiparametric_2017, buchanan_quantitative_2019}. This protocol is used to measure multiple quantitative properties of the kidneys with relative increases/decreases between measurements functioning as biomarkers and therefore indications of \ac{CKD} progression. The implementation of new quantitative renal imaging methods can improve this protocol, thus increasing its clinical application. In addition to the \ac{CKD} paradigm, we wish to apply these methods ex-vivo, both to allow a more direct comparison with current gold standards, such as histopathology, and to aid with assessment of renal allograft viability.

\section{Thesis Overview}
%\addcontentsline{toc}{section}{Thesis Overview}
\doublecheck{Make sure these are in the correct order}

\textbf{Chapter \ref{chap:Theory}} provides the theoretical framework of \ac{NMR} and \ac{MRI}. A detailed description if given of the origin of the measured signal, processes that give rise to contrast between tissues and the methods of image formation. \\

\textbf{Chapter \ref{chap:t2_mapping}} explores \ttwo mapping within the kidneys. There is little consensus as to which method should be used within the kidneys \cite{dekkers_consensus-based_2019}, thus leading to inconsistent values quoted between studies \cite{wolf_magnetic_2018}. Here multiple methods from the literature are compared assessing their quantitative accuracy, sensitivity to flow and image quality in phantoms before five subjects are scanned to assess the methods in-vivo.\\

\textbf{Chapter \ref{chap:TRUST}} aims to translate methods for measuring blood oxygenation from vessels in the brain to use within the kidneys. Focusing on \ac{SBO} \cite{jain_mri_2010} and \ac{TRUST} \cite{lu_quantitative_2008} this chapter optimises the methods for use in the abdomen, verifying the modifications in the brain, then carries out an oxygen challenge in-vivo to measure changes in oxygen saturation within the renal vein.\\

\textbf{Chapter \ref{chap:ML}} describes the development of a fully automated method to segment the kidneys from \ac{MRI} data. Defining renal masks is an important, yet time consuming, aspect of many studies. The masks can be used to calculate \ac{TKV} or to inform downstream processing. Here a \ac{CNN} is developed to segment the kidneys from \ttwo weighted \ac{HASTE} images. Software is developed to provide an executable that allows anyone to segment the kidneys in a few seconds on regular office hardware.\\

\textbf{Chapter \ref{chap:Neph}} develops methods for scanning kidneys ex-vivo. The clinical gold standard for diagnosis of renal pathologies is biopsy followed by histological analysis. Comparison between this gold standard and recently developed quantitative \ac{MRI} techniques is vital for clinical translation. Here a pipeline for multi-parametric imaging of the same kidney in-vivo, ex-vivo followed by histology is developed.\\

\textbf{Chapter \ref{chap:conclusion}} concludes the thesis, highlighting key results and their current applications. It also provides an overview as to future research directions and how the methods developed could be applied to new paradigms or expanded upon.\\

\newpage
\section{References}
%\addcontentsline{toc}{section}{References}
\defbibheading{bibliography}[\refname]{}
\printbibliography