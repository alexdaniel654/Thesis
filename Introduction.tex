\chapter*{Introduction}
\addcontentsline{toc}{chapter}{Introduction}
From April 2019 to March 2020, the United Kingdom's \ac{NHS} performed 44,884,450 medical imaging procedures, of these 3,811,415 were \ac{MRI} \cite{noauthor_diagnostic_2020}. This technique can be used to produce high resolution volumetric images of the body with exquisite soft tissue contrast. Unlike other modalities, such as \ac{CT} and \ac{PET}, \ac{MRI} uses non-ionising radiation, making it more suitable for longitudinal analysis of patient progression and research involving healthy volunteers.\\

The superior soft tissue contrast of \ac{MRI} compared to \ac{CT} meant it first found widespread clinical adoption in the field of neuroimaging. Here \ac{MRI} has been used for diagnosis of neurological disorders, monitoring treatment progression and research into cognition. Many of the techniques honed in the brain, can be applied to the abdomen, where similar tissue properties can exploit the same techniques; albeit in a somewhat more challenging environment due to respiratory motion and a more inhomogeneous tissue structure. The kidneys are ideally suited to this translation as they have similar tissue properties to the brain and are highly dynamic organs.\\

In addition to the acquisition of basic structural images, \ac{MRI} can be used to collect quantitative information about the tissues being imaged. In this situation the numerical voxel values have physical significance, rather than simply representing signal intensity in arbitrary units \cite{tofts_quantitative_2003}. Using quantitative \ac{MRI} properties such as an organs oxygen consumption \cite{zhang_quantitative_2015}, perfusion \cite{karger_quantitation_2000}, stiffness \cite{mariappan_magnetic_2010} and temperature \cite{yuan_towards_2012} can be measured. Although many quantitative \ac{MRI} techniques have been developed for the the kidneys, there is still many methods where development, translation from the brain or standardisation with the wider renal community would be highly desirable.

\section*{Clinical Motivation}
\addcontentsline{toc}{section}{Clinical Motivation}
Kidneys; wot do? \todo{Add a little bit about renal anatomy, CKD and maybe transplants.}

\section*{Thesis Overview}
\addcontentsline{toc}{section}{Thesis Overview}
\doublecheck{Make sure these are in the correct order}

\textbf{Chapter \ref{chap:Theory}} provides the theoretical framework of \ac{NMR} and \ac{MRI}. A detailed description if given of the origin of the measured signal, processes that give rise to contrast between tissues and the methods of image formation. \\

\textbf{Chapter \ref{chap:t2_mapping}} explores \ttwo mapping within the kidneys. There is little consensus as to which method should be used within the kidneys \cite{dekkers_consensus-based_2019}, thus leading to inconsistent values quoted between studies \cite{wolf_magnetic_2018}. Here multiple methods from the literature are compared assessing their quantitative accuracy, sensitivity to flow and image quality in phantoms before five subjects are scanned to assess the methods in-vivo.\\

\textbf{Chapter \ref{chap:TRUST}} aims to translate methods for measuring blood oxygenation from vessels in the brain to use within the kidneys. Focusing on \ac{SBO} \cite{jain_mri_2010} and \ac{TRUST} \cite{lu_quantitative_2008} this chapter optimises the methods for use in the abdomen, verifying the modifications in the brain, then carries out an oxygen challenge in-vivo to measure changes in oxygen saturation within the renal vein.\\

\textbf{Chapter \ref{chap:ML}} describes the development of a fully automated method to segment the kidneys from \ac{MRI} data. Defining renal masks is an important, yet time consuming, aspect of many studies. The masks can be used to calculate \ac{TKV} or to inform downstream processing. Here a \ac{CNN} is developed to segment the kidneys from \ttwo weighted \ac{HASTE} images. Software is developed to provide an executable that allows anyone to segment the kidneys in a few seconds on regular office hardware.\\

\textbf{Chapter \ref{chap:Neph}} develops methods for scanning kidneys ex-vivo. The clinical gold standard for diagnosis of renal pathologies is biopsy followed by histological analysis. Comparison between this gold standard and recently developed quantitative \ac{MRI} techniques is vital for clinical translation. Here a pipeline for multi-parametric imaging of the same kidney in-vivo, ex-vivo followed by histology is developed.\\

\textbf{Chapter \ref{chap:conclusion}} concludes the thesis, highlighting key results and their current applications. It also provides an overview as to future research directions and how the methods developed could be applied to new paradigms or expanded upon.\\

\newpage
\section*{References}
\addcontentsline{toc}{section}{References}
\defbibheading{bibliography}[\refname]{}
\printbibliography