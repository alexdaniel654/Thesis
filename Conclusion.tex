\chapter{Conclusion}
\label{chap:conclusion}
\newpage
The work presented in this thesis has developed techniques for quantitative renal \ac{MRI}. These techniques will complement existing multiparametic renal protocols, Figure \ref{fig:conc_multi_para_overview}.

\begin{figure}[H]
	\centering
	\includegraphics[width=1\textwidth]{Intro/Multiparametric_Renal_MRI_TRUST.eps}
	\caption{An overview of the multiparametric renal \ac{MRI} protocol. This shows how the areas focused on in this thesis (highlighted in orange) fit into the full protocol designed to provide a complete evaluation of renal health using \ac{MRI}.}
	\label{fig:conc_multi_para_overview}	
\end{figure}

In Chapter \ref{chap:t2_mapping}, the four commonly used renal \ttwo mapping sequences of \ac{SE}-\ac{EPI}, \ac{ME-TSE}, \ac{GraSE} and \ac{CPMG} \ttwo preparation, were methodically evaluated. In addition to the \ttwo mapping acquisition schemes, methods of calculating \ttwo maps from resulting multi-echo data were compared. This comparison was first performed using multiple calibrated phantoms to assess accuracy, including sensitivity to flow, and spatial blurring. Each method was then used to collect \ttwo maps of five healthy volunteers to evaluate their performance in-vivo. 

The \ac{SE}-\ac{EPI} sequence was the most accurate over the range of \ttwo found in the kidneys (40 - 200~ms) with a \ac{MPE} of 8~$\pm$~5~\% in the static phantom however this sequence was also highly sensitive to flow. This is sub-optimal for renal \ttwo mapping given the kidneys are highly perfused and the effects of flow of water-like filtrates through renal tubules can clearly be seen in-vivo. The \ac{ME-TSE} sequence was the least accurate of the sequences evaluated in the static phantom (\ac{MPE} of 23~$\pm$~13~\%) however its sensitivity to flow and degree of spatial blurring were an improvement on those of the \ac{SE}-\ac{EPI}. The \ac{GraSE} sequence produced a \ac{MPE} of 15~$\pm$~4~\% in the static phantom and was the least sensitive to flow of the sequences compared. In-vivo it showed the greatest contrast between cortex and medulla. The \ac{CPMG} \ttwo prep sequence yielded a \ac{MPE} of 11~$\pm$~1~\% and was relatively insensitive to flow however suffered from a large degree of spatial blurring thus limiting its in-vivo utility. It was concluded that the \ac{GraSE} sequence is the optimum protocol for renal \ttwo mapping on the Philips platform due to its combination of quantitative accuracy, insensitivity to flow and superior image quality. For the kidneys, a basic two parameter fitting model produces the most accurate \ttwo map of the fitting method compared. In future these techniques should be applied to study renal disease patients to assess inflammation of tissues. Here the \ac{GraSE} sequence was optimised for the Philips platform; for multicentre studies the availability of sequences across vendors must be considered. The \ac{ME-TSE} sequence is more commonly implemented across vendors and as such may be preferable for multicentre studies.

Chapter \ref{chap:TRUST} focused on the translation of methods for measuring blood oxygen saturation from the superior sagittal sinus in the brain to the renal vein. Currently renal oxygenation is typically assessed using \ac{BOLD} \ttwostar maps, however, these maps are influenced by factors other than tissue oxygenation such as susceptibility effects, shimming and baseline blood flow. As such, it is desirable to have an alternative method to measuring renal oxygenation. \ac{SBO} was performed and \ac{TRUST} optimised for use in the abdomen. The \ac{TRUST} optimisation involved replacing the \ac{TILT} labelling scheme with a \ac{FAIR} labelling scheme, applying respiratory triggering and optimising the \ac{PLD} for the renal vessels. \ac{SBO} was found to be unsuitable for use in the kidneys due to the geometry of the angle of the renal vein to the $B_0$ field. \ac{TRUST} was successfully used to measure the oxygen saturation of the portal vein and hepatic artery and measured an increase in oxygen saturation in the renal vein of $16 \pm 3~\%$ during an oxygen challenge.

The \ac{TRUST} scheme is currently being applied in the study of \textsc{covid}-19 in which 35~\% of patients experience acute kidney injury for which hypoxia and altered blood flow are thought to be key determinants. Data is demonstrating a reduction in venous oxygenation from 86~$\pm$~7~\% to 59~$\pm$~8~\% in acute \ac{ICU} ventilated \textsc{covid}-19 patients. This work is being performed in collaboration with Uppsala University. The use of \ac{TRUST} to study oxygenation in the liver circulation within patient populations is also being explored in collaboration with Dr Bryan Haddock of Rigshospitalet, Copenhagen University Hospital.

Machine learning methods were used in Chapter \ref{chap:ML} to segment the kidneys from \ttwo-weighted \ac{HASTE} images to calculate \ac{TKV}. A \ac{CNN} was trained on images from both healthy volunteers and \ac{CKD} patients and used to predict the \ac{TKV} of ten unseen subjects, each scanned five times. This resulted in a mean Dice score of 0.93~$\pm$~0.01 and a mean volume difference from manual segmentations of 1.2~$\pm$~16.2~m$\ell$. The \ac{CNN} also produced a significantly lower \ac{CoV} than the human observers (1.5~$\pm$~0.5~\% and 2.7~$\pm$~0.9~\% respectively, p~=~0.008). A self contained executable has been produced to enable use of the \ac{CNN} by inexperienced users to reduce segmentation times from 15 - 30 minutes per subject to approximately 10 seconds. In future work the performance of the network will be evaluated across multiple vendors (GE, Philips and Siemens,) and additional architectures will be explored, initially by including adjacent slices as colour channels and later expanding to a full 3D \ac{CNN}. 

In Chapter \ref{chap:ex} a multiparametric ex-vivo protocol is developed to enable the assessment of explanted renal samples. This protocol involved the acquisition of relaxation maps and diffusion based measures ex-vivo, for each an in-vivo counterpart is also demonstrated. The aim for the future is to apply these ex-vivo protocols and in-vivo measures in a nephrectomy model where the ex-vivo measures can be used with existing histopathological processing pipelines to better understand the interplay between quantitative \ac{MRI} parameters and underlying histological processes. Additionally the ex-vivo paradigm is an excellent foundation for future developments of \ac{MRI} based nephron number measurements utilising super resolution techniques to upsample data beyond the acquisition resolution.

Further, the high spatial resolution, statistical nature of ex-vivo data has been used to validate a method of assessing the change in quantitative \ac{MRI} measures with depth of renal tissue. This method was implemented using a bespoke Freesurfer and \textsc{matlab} pipeline to generate quantitative maps of tissue depth. This pipeline can be applied to any quantitative data in the multiparametric protocol provided a full organ coverage anatomical scan is acquired in the same session. It avoids the need to adjust quantitative acquisition protocols and allows an additional analysis method for renal \ac{MRI} data to augment the long-established practice of cortical and medullary segmentation.

Moving forward, the techniques developed in this thesis can be incorporated into multiparametric renal \ac{MRI} studies undertaken at the \ac{SPMIC} and more widely across the \ac{UKRIN} and Parenchima \ac{COST} initiatives. Current areas of active research in patient populations at the \ac{SPMIC} focus on \acl{CKD}, \ac{AKI}, the effects of dialysis and \textsc{covid}-19. Additionally, the ex-vivo protocol could be used for the assessment of renal transplants, increasing confidence in organ quality prior to transplant and thus enabling an increased use of kidneys currently deemed marginal in quality. In the long term, this could be used as a rapid scan of allografts in transplant centres to increase transplant rates and therefore reduce waiting list times. 