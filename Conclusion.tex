\chapter{Conclusion}
\label{chap:conclusion}
\newpage
The work presented in this thesis has developed techniques for quantitative renal \ac{MRI}. These techniques will complement existing multiparametic renal protocols.

In Chapter \ref{chap:t2_mapping} the four most commonly used renal \ttwo mapping sequences were methodically evaluated. In addition to the acquisition schemes, methods of calculating \ttwo maps from resulting multi-echo data were also compared. This comparison begin with of a thorough assessment of each techniques quantitative accuracy using multiple calibrated phantoms. Each method was then used to collect \ttwo maps of five healthy volunteers to evaluate their performance in-vivo. It was concluded that the \ac{GraSE} sequence is the optimum protocol for renal \ttwo mapping due to its combination of quantiative accuracy, insensitivity to flow and superior image quality. A basic two parameter fitting model produces the most accurate \ttwo map of the fitting method compared.

Chapter \ref{chap:TRUST} focused on the translation of methods for measuring blood oxygen saturation from the superior sagittal sinus in the brain, to the renal vein. \ac{SBO} and \ac{TRUST} were both optimised for use in the abdomen with all modifications being verified in the brain. \ac{SBO} was found to be unsuitable for use in the kidneys due to the geometry of the angle of the renal vein to the $B_0$ field. \ac{TRUST} was successfully used to measure an increase in venous oxygen saturation of $16 \pm 3 \%$ during an oxygen challenge.

Machine learning methods were used in Chapter \ref{chap:ML} to segment the kidneys from \ttwo weighted \ac{HASTE} images and calculate \ac{TKV}. A \ac{CNN} was trained on images from both healthy volunteers and \ac{CKD} patients and used to predict the \ac{TKV} of ten unseen subjects, each scanned five times. This resulted in a mean Dice score of 0.93 $\pm$ 0.01 and a mean volume difference from manual segmentations of 1.2 $\pm$ 16.2 m$\ell$. The \ac{CNN} also produced a significantly lower \ac{CoV} than the human observers (1.5 $\pm$ 0.5 \% and 2.7 $\pm$ 0.9 \% respectively, p = 0.008). A self contained executable has been produced to make using the \ac{CNN} suitable for operation by inexperience users and reduced segmentation times from 15 - 30 minutes per subject to approximately 10 seconds.

Many of the methods outlined in previous chapters were utilised in the development of a multiparametric ex-vivo protocol to enable the assessment of explanted renal samples in Chapter \ref{chap:ex}. This protocol involves the acquisition of \tone, \ttwo, \ttwostar, \ac{ADC} and \ac{FA} and tractography both ex-vivo and in-vivo and can be used with existing histopathological processing pipelines to better understand the interplay between quantitative \ac{MRI} parameters and underlying histological processes. 

Moving forward, the techniques developed in this thesis can be incorporated into multiparametric renal \ac{MRI} studies undertaken at \ac{SPMIC}. Current areas of active research in patient populations focus on \ac{CKD}, \ac{AKI}, the effects of dialysis and \textsc{covid}-19. Additionally, the ex-vivo protocol could be used to for the assessment of renal transplants, increasing confidence in organ quality prior to transplant and thus enabling an increased use of kidneys currently deemed marginal in quality. This could increase transplant rates and therefore reduce waiting list times.