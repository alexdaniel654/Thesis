\chapter{Publications}

The following work was submitted to and presented at the \ac{ISMRM} 27th Annual Meeting.

\section*{The Effects of Fixation and Age on Relaxometry Measurements of Ex-Vivo Kidneys}
\label{sec:ISMRM_Neph}
\subsection*{Synopsis}
The study of post-mortem brain tissue using MRI has been shown to provide a tool to assess whole organ microstructure and pathology with high spatial resolution. However, few studies have been performed on other organs in the body, here we perform ex-vivo imaging of whole kidneys. $T_1$ and $T_2^*$ of ex-vivo porcine kidneys are monitored over a ten-week period to study how $T_1$ and $T_2^*$ of the renal cortex and medulla vary over time from fixation. A clear understanding of the effects of fixation on tissue MRI parameters is crucial for interpreting ex-vivo MRI studies.

\subsection*{Purpose}

Renal pathologies are currently assessed via biopsy followed by histological staining, this process is invasive and not representative of the entire kidney. Recently, multiparametric renal MRI protocols \cite{cox_multiparametric_2017} have been suggested to provide measures of underlying pathology \cite{friedli_new_2016}, for example longitudinal relaxation time ($T_1$) and apparent diffusion coefficient (ADC) have been suggested to be markers of inflammation and fibrosis. Ex-vivo MRI can offer valuable quantitative measures for validating in-vivo MRI by comparing whole organ histology with high spatial-resolution and signal-to-noise ratio MRI data. By studying subjects who are undergoing a nephrectomy, the kidney can be scanned in-vivo, then upon removal the whole kidney can be scanned ex-vivo and biopsied for histology providing three complimentary streams of data. 

The study of post-mortem tissue samples has provided a better understanding of brain pathology from novel features of  tissue architecture, relaxometry methods ($T_1$, $T_2$ and $T_2^*$), and diffusion tensor imaging (DTI) \cite{birkl_effects_2016, kolasinski_combined_2012, miller_diffusion_2011}. However most post-mortem studies involve immersion fixing with formalin. The effects of fixation on brain tissue has been reported \cite{tovi_measurements_1992, shatil_quantitative_2018} but no literature is available on the effect of fixation on renal tissue. Here we characterize the changes of different MRI properties in the porcine kidney, whose renal anatomy is very similar to humans, to facilitate future ex-vivo relaxometry and DTI studies of the human kidney.

\subsection*{Methods}
\subsubsection*{Data Acquisition}

Whole porcine kidneys were fixed by placing them into ten-times their volume of 10\% Neutral Buffered Formalin for 24-hours. They were then transferred to six times their volume of Phosphate-buffered Saline (PBS) to wash out excess formalin and rehydrate the kidneys \cite{sengupta_high_2017}. To study the changes in $T_1$ following removal from formalin, the kidneys were (i) scanned regularly over the first 24-hours in PBS, and (ii) scanned over a ten-week period in PBS. In addition, an unfixed kidney was scanned. Kidneys were scanned on both a 3T Philips Ingenia and 7T Philips Achieva to collect $T_1$ and $T_2^*$ relaxometry maps. $T_1$ and $T_2^*$ maps were generated using an ultrafast gradient echo scheme and a multi-shot FFE sequence respectively (Fig \ref{fig:ISMRM_Fig_1}). To further study the renal inflammation and fibrosis, known aged pigs were euthanized and kidneys scanned after 24-hours in PBS and in addition histology was performed on the renal cortex using Haemotoxylin and Eosin (H\&E) and Masson’s trichrome stains, to date kidneys have been collected from a 0.5-year and 2.5-year pig.

\begin{figure}[H]
	\centering
	\includegraphics[width=0.6\textwidth]{Files/ISMRM/Table_1_crop.png}
	\caption{The imaging parameters used to acquire $T_1$ and $T_2^*$ maps.}
	\label{fig:ISMRM_Fig_1}
\end{figure}

\subsubsection*{Data Analysis}

$T_1$ maps were formed by fitting the magnitude data corrected using the phase information \cite{szumowski_signal_2012} on a voxel-by-voxel basis to an inversion recovery. $T_2^*$ maps were generated by using a weighted echo time fit from the log of the exponential signal decay. Once quantitative maps were made, regions of interest were defined for renal cortex and medulla.

\subsection*{Results}

Example high resolution whole kidney $T_1$ and $T_2^*$ maps are shown in Fig. \ref{fig:ISMRM_Fig_2} The $T_1$ and $T_2^*$ maps were imaged over 24-hours and ten-weeks (Fig. \ref{fig:ISMRM_Fig_3}). There was little change in $T_1$ over 24-hours of immersion in PBS (Fig.\ref{fig:ISMRM_Fig_3}b), and $T_1$ was close to that of an unfixed kidney. There was a large change in $T_1$ observed between 24-hours and one week, after this there was a general decline in $T_1$ of the medulla while the cortex remained constant (Fig.\ref{fig:ISMRM_Fig_3}b). The $T_2^*$ of the medulla remains constant over the ten-weeks whilst the renal cortex $T_2^*$ increases. Figure \ref{fig:ISMRM_Fig_4} shows matched slices from the 3T $T_1$, 7T $T_1$ and 7T $T_2^*$ maps. In Fig.\ref{fig:ISMRM_Fig_4}b it is possible to differentiate between the cortex and outer medulla at 24-hours, but after one week this differentiation is no longer visible. No small scale changes are visible in the $T_2^*$ maps, only the bulk increase in $T_2^*$ of the cortex. Fig. \ref{fig:ISMRM_Fig_5} shows 3T $T_1$ and $T_2^*$ maps of the 0.5 and 2.5 year-old kidneys and associated measures in renal cortex and medulla. MR data shows no significant difference in $T_1$ or $T_2^*$ of the cortex between 0.5 and 2.5 years; both $T_1$ and $T_2^*$ of the medulla is lower in 2.5-year kidney. No significant differences were seen between the histology of these pigs. Using both MRI and histology, no difference in cortical tissue was observed between 0.5 and 2.5-year pigs, we now plan to scan older pig kidneys which will have established levels of fibrosis. 

\begin{figure}[H]
	\centering
	\includegraphics[width=0.6\textwidth]{Files/ISMRM/Figure_2.eps}
	\caption{Example $T_1$ and $T_2^*$ maps generated 24-hours after fixation. a. $T_1$ measured at 3T b. $T_1$ measured at 7T c. $T_2^*$ measured at 3T d. $T_2^*$ measured at 7T.}
	\label{fig:ISMRM_Fig_2}
\end{figure}

\begin{figure}[H]
	\centering
	\includegraphics[width=0.8\textwidth]{Files/ISMRM/Figure_3.eps}
	\caption{a. The $T_1$ map of an unfixed kidney at 3T b. The change in $T_1$ measured at 3T over 24-hours. c. The change in $T_1$ and $T_2^*$ measured at 3T and 7T.
	}
	\label{fig:ISMRM_Fig_3}
\end{figure}

\begin{figure}[H]
	\centering
	\includegraphics[width=0.6\textwidth]{Files/ISMRM/Figure_4.eps}
	\caption{Matched slices from the a. 3T $T_1$, b. 7T $T_1$ and c. 7T $T_2^*$ maps at 24-hours, one week and six weeks. The boundary between the cortex and outer medulla is clearly visible in the 7T $T_1$ map at 24-hours, but less apparent at 1 and 6 weeks.}
	\label{fig:ISMRM_Fig_4}
\end{figure}

\begin{figure}[H]
	\centering
	\includegraphics[width=0.9\textwidth]{Files/ISMRM/Figure_5.eps}
	\caption{ Data collected from 0.5 year-old and 2.5 year-old pig kidney. a. $T_1$ maps collected at 3T b.  H\&E stained sample of cortex c. Masson trichrome stained sample of cortex. d. Renal cortex and medulla $T_1$ for the 0.5 year-old and 2.5 year old kidneys, and renal cortex and medulla $T_2^*$ for the 0.5 year-old and 2.5 year-old kidneys. }
	\label{fig:ISMRM_Fig_5}
\end{figure}

\subsection*{Conclusion}

Here we show that by scanning kidney samples for ten-weeks post fixation, $T_1$ is similar to an unfixed kidney in the first 24-hours, after this $T_1$ and $T_2^*$ have a dependence on time after fixation. This suggests that nephrectomy samples should be scanned within 24-hours of rehydration.
