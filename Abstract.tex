\begin{abstract}
	\addcontentsline{toc}{chapter}{Abstract}
	The kidneys are morphologically and functionally complex organs and as such, lend themselves to complex methodologies of study. One such methodology is quantitative \ac{MRI}. Rather than simply taking a structural image of the kidneys, quantitative \ac{MRI} aims to measure physical properties such as the rate of blood flow, tissue perfusion, oxygen consumption and more fundamental properties of the matter making up the organ such as its proton density or longitudinal relaxation time, \tone and transverse relaxation time \ttwo. This is done without the need for ionising radiation and often without exogenous contrast agents, thus making \ac{MRI} an ideal tool for both clinical and research use.
	
	Multiple methods have been developed to measure the transverse relaxation time, \ttwo, of the kidneys, often leading to inconsistent results between studies. Here, a methodical comparison of four prominent techniques is performed. This comparison makes use of quantitative phantoms before proceeding to assess each technique in-vivo in healthy volunteers. A \ac{GraSE} sequence is recommended for future renal \ttwo mapping.
	
	Techniques to measure the \ac{RMRO$_2$} would be highly desirable. \ac{SBO} and \ac{TRUST} are modified for use in the abdomen. \ac{SBO} is found to be poorly suited to measuring oxygenation in the renal veins, however \ac{TRUST} is used to successfully measure changes in venous oxygenation in the renal vein during an oxygen challenge.
	
	Manual definition of the kidneys to compute \ac{TKV} is a tedious and labour intensive bottleneck in many renal \ac{MRI} studies. Here a \ac{CNN} is developed to generate fully automated masks of the kidneys to compute \ac{TKV} with better than human precision.
	
	Finally, quantitative renal mapping methods are developed for an ex-vivo renal \ac{MRI} protocol to enable future correlation with histopathology pipelines. Correlating these two diagnostic methods should aid clinical adoption of renal \ac{MRI}, increase confidence in diagnostics, improved patient experience, and will have applications in nephrectomy studies and transplantation.
\end{abstract}